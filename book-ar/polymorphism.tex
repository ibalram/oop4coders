%=====================================================================
\ifx\wholebook\relax\else
	\documentclass{KodeBook}
	\usepackage{amsmath,amssymb}             % AMS Math
\usepackage[utf8]{inputenc}
%\usepackage[T1]{fontenc}
\usepackage{hyperref}
%\usepackage[pdftex]{graphicx}
\usepackage{listings}
%\usepackage{listingsutf8}
%\usepackage{xcolor}
%\usepackage{times}
\usepackage{array}
%\usepackage{natbib}
%\usepackage[left=2.8cm,right=2.2cm,top=2.8cm,bottom=2.8cm,includefoot,includehead,headheight=13.6pt]{geometry}

%\bibliographystyle{ACM-Reference-Format-Journals}


\lstdefinestyle{codeStyle}{
  belowcaptionskip=1\baselineskip,
  breaklines=true,
  frame=L,
  xleftmargin=1.5cm,%\parindent,
  showstringspaces=false,
  basicstyle=\scriptsize\ttfamily\bfseries,
  keywordstyle=\bfseries\color{green!40!black},
  commentstyle=\itshape\color{purple!40!black},
  identifierstyle=\color{blue},
  stringstyle=\color{orange},
  backgroundcolor=\color{blue!10!white},
  lineskip=.2em,
  numbers=left
%  frameround=tttt
}

\lstdefinestyle{shellStyle}{
	belowcaptionskip=1\baselineskip,
	breaklines=true,
	frame=shadwbox,
	rulesepcolor=\color{blue},
	xleftmargin=1.25cm,
	basicstyle=\scriptsize\ttfamily\bfseries\color{green},
	backgroundcolor=\color{black},
	lineskip=.2em,
%	morekeywords={sudo},keywordstyle=\color{red},
	%  frameround=tttt
}



\hypersetup{
	colorlinks,
	urlcolor=blue
}

\renewcommand{\UrlFont}{\ttfamily\footnotesize}

	\begin{document}
\fi
%=====================================================================

\chapter{\textarab{التعددية الشكلية}}

\begin{introduction}
	\lettrine{P}{olymorphism} is another key concept of \ac{oop}. 
	In general, the term refers to the ability to assign a different meaning or usage to something in many contexts.
	Hence, it is not a concept specific to \ac{oop} since you can find other forms such as parametric polymorphism (generic programming). 
	In languages with type inheritance, an object of a class can be considered as an object of its superclass; and therefore, the same applications can be executed on it.
	Some languages afford functionalities to verify if a certain object is instantiated from a class, to get its class or to cast the variable using it to another subclass.
	Methods can express polymorphism in two ways: overloading (same name with multiple signatures) and overriding (same signature, different implementation). 
	Not every language affords overloading, but methods overriding is a core task of \ac{oop}.
\end{introduction} 

\section{Subtype polymorphism}

According to the \nameword{Liskov substitution principle} \citep{1987-liskov}, a function written to take an object of a certain type \textbf{T} must work correctly if passed an object of a type \textbf{S} which is a subtype of \textbf{T}.
C++ and Java are examples of \ac{oop} languages with static type checking affording this principle. 
Some languages do not need an object to be created from a class/prototype extending another to execute certain treatments on it. 
They use a mechanism called \nameword{Duck typing} which states that ``\textit{If it walks like a duck and talks like a duck, it must be a duck}". 

A simple example will be sufficient to show how every language handles such polymorphism. 
Having a class \textbf{Person} with a method \textbf{talk} and two subclasses \textbf{Student} and \textbf{Professor}:
\begin{itemize}
	\item A function \textbf{announce} designed to call \textbf{talk} of \textbf{Person} is passed objects of \textbf{Student} and \textbf{Professor} as argument.
	\item An array intended to contain objects of \textbf{Person} is filled with objects of its subtypes.
	\item If the language supports Duck typing: some unrelated classes and objects having the method \textbf{talk} can be used with the two previous propositions.
\end{itemize}

\subsection{C++}

Lets create our classes

\lstinputlisting[language={[KB]C++}, linerange={4-13}, style=codeStyle]{../codes/cpp/polymorphism/subtype.cpp}

Passing objects of these three classes to a function designed for objects of type \textbf{Person} will work just fine. 

\lstinputlisting[language={[KB]C++}, linerange={15-26, 31-33}, style=codeStyle]{../codes/cpp/polymorphism/subtype.cpp}

Objects of type \textbf{Person} or any other subclass can be assigned to a variable of type \textbf{Person}.

\lstinputlisting[language={[KB]C++}, linerange={20-23, 28-33}, style=codeStyle]{../codes/cpp/polymorphism/subtype.cpp}

\subsection{Java}

Lets create our classes

\lstinputlisting[language={[KB]Java}, linerange={35-40,45-47}, style=codeStyle]{../codes/java/src/polymorphism/Subtype.java}

Passing objects of these three classes to a function designed for objects of type \textbf{Person} will work just fine. 

\lstinputlisting[language={[KB]Java}, linerange={3-7, 13-20, 32-33}, style=codeStyle]{../codes/java/src/polymorphism/Subtype.java}

If we use an interface (\textbf{Machine}) and a function \textbf{announce2} similar to \textbf{announce} but taking as parameter \textbf{Machine} instead of \textbf{Person}, this will work as well because interfaces in Java define new types and support type polymorphism.

\lstinputlisting[language={[KB]Java}, linerange={3-3, 8-8, 13-13, 25-34, 40-44, 48-55}, style=codeStyle]{../codes/java/src/polymorphism/Subtype.java}

Objects of type \textbf{Person} or any other subclass can be assigned to a variable of type \textbf{Person}.

\lstinputlisting[language={[KB]Java}, linerange={3-3, 13-16,22-24,32-33}, style=codeStyle]{../codes/java/src/polymorphism/Subtype.java}

\subsection{Javascript}

Lets create our classes (\ac{es6} style, \ac{es5} code is afforded as well): all of them afford a method \textbf{talk}.

\lstinputlisting[language={[KB]Javascript}, linerange={3-17}, style=codeStyle]{../codes/javascript/polymorphism/es6/subtype.js}

A javascript function does not specify the types of its parameters. 
In this function, it is clear that the object named \textbf{talker} must define a method called \textbf{talk}. 
Because we do not know what type the object is, and if it has a function \textbf{talk}, we can verify before calling it.

\lstinputlisting[language={[KB]Javascript}, linerange={19-23}, style=codeStyle]{../codes/javascript/polymorphism/es6/subtype.js}

Then, objects created from class \textbf{Person} and its subclasses can be passed to \textbf{announce} since they all afford the method \textbf{talk}. 
Also, objects created from a different class (\textbf{Robot}) or created directly containing that method can be passed as well, without causing any problem.
If an object has no method \textbf{talk} (lets say: a number), the function \textbf{announce} will print the other message instead of calling our method.

\lstinputlisting[language={[KB]Javascript}, linerange={25-41}, style=codeStyle]{../codes/javascript/polymorphism/es6/subtype.js}

Lets try an array with different elements 

\lstinputlisting[language={[KB]Javascript}, linerange={43-52}, style=codeStyle]{../codes/javascript/polymorphism/es6/subtype.js}

\subsection{Lua}

Lets create our classes: all of them afford a method \textbf{talk}.
In this code, only the superclass affords the function \textbf{new} by assigning \keyword{self} (the class name) as meta-table of the newly created object(table).  
The benefit of doing this is to not repeat the same \textbf{new} method for subclasses.

\lstinputlisting[language={[KB]Lua}, linerange={1-23}, style=codeStyle]{../codes/lua/polymorphism/subtype.lua}

A lua function does not specify the types of its parameters. 
In \textbf{announce}, it is clear that the object named \textbf{talker} must define a method called \textbf{talk}. 
Because we do not know what type the object is, and if it has a function \textbf{talk}, we can verify if it is a table and if it has a member of type \textbf{function} called \textbf{talk}.

\lstinputlisting[language={[KB]Lua}, linerange={25-32}, style=codeStyle]{../codes/lua/polymorphism/subtype.lua}

Then, objects created from class \textbf{Person} and its subclasses can be passed to \textbf{announce} since they all afford the method \textbf{talk}. 
Also, objects created directly using a table affording the same method can be passed without problem.

\lstinputlisting[language={[KB]Lua}, linerange={34-46}, style=codeStyle]{../codes/lua/polymorphism/subtype.lua}

Lets try an array with different elements 

\lstinputlisting[language={[KB]Lua}, linerange={48-55}, style=codeStyle]{../codes/lua/polymorphism/subtype.lua}

\subsection{Perl}

Lets create our classes: all of them afford a method \textbf{talk} even the one not inheriting from \textbf{Person}.
The \textbf{new} method blesses a hash (our object to be) to the first argument which is the class calling it. 
When other classes inherit \textbf{new}, each will pass its name when calling this method; and therefore, the new object will be blessed to it.

\lstinputlisting[language={[KB]Perl}, linerange={1-17}, style=codeStyle]{../codes/perl/polymorphism/subtype.pl}

A perl subroutine does not specify the types of its parameters and either their number.
In \textbf{announce}, it is clear that the object named \textbf{talker} must define a method called \textbf{talk}. 
Because we do not know what type the object is, we can verify if it has a method called \textbf{talk} using the UNIVERSAL method \keyword[Perl]{can}.

\lstinputlisting[language={[KB]Perl}, linerange={19-24}, style=codeStyle]{../codes/perl/polymorphism/subtype.pl}

Any object created from a class affording \textbf{talk}, or one of its subclasses can be passed to the function \textbf{announce}. 

\lstinputlisting[language={[KB]Perl}, linerange={26-35}, style=codeStyle]{../codes/perl/polymorphism/subtype.pl}

Lets try an array with different elements 

\lstinputlisting[language={[KB]Perl}, linerange={37-42}, style=codeStyle]{../codes/perl/polymorphism/subtype.pl}

\subsection{PHP}

Lets create our classes: all of them afford a method \textbf{talk} either directly or by inheritance.

\lstinputlisting[language={[KB]PHP}, linerange={3-17}, style=codeStyle]{../codes/php/polymorphism/subtype.php}

A PHP function does not specify the types of its parameters. 
In \textbf{announce}, it is clear that the object named \textbf{talker} must define a method called \textbf{talk}. 
So, we have to verify that using a function called \keyword[PHP]{method\_exists}. 

\lstinputlisting[language={[KB]PHP}, linerange={19-23}, style=codeStyle]{../codes/php/polymorphism/subtype.php}

Then, objects created from any class affording the method \textbf{talk} can be passed to \textbf{announce} without a problem. 
If an object has no method \textbf{talk} (lets say: a number), the function \textbf{announce} will print the other message instead of calling our method.

\lstinputlisting[language={[KB]PHP}, linerange={25-34}, style=codeStyle]{../codes/php/polymorphism/subtype.php}

Lets try an array with different elements 

\lstinputlisting[language={[KB]PHP}, linerange={36-45}, style=codeStyle]{../codes/php/polymorphism/subtype.php}

\subsection{Python}

Lets create our classes: all of them afford a method \textbf{talk} either directly or by inheritance.

\lstinputlisting[language={[KB]Python}, linerange={4-14}, style=codeStyle]{../codes/python/polymorphism/subtype.py}

A Python function does not specify the types of its parameters. 
In \textbf{announce}, it is clear that the object named \textbf{talker} must define a method called \textbf{talk}. 
So, we have to retrieve that method using a function called \keyword[Python]{getattr}; if there is no member called \textbf{talk}, the function will return \keyword[Python]{None} as specified in the code.
To verify if it is a method and not a field, we use the function \keyword[Python]{callable}.

\lstinputlisting[language={[KB]Python}, linerange={16-23}, style=codeStyle]{../codes/python/polymorphism/subtype.py}

Then, objects created from any class affording the method \textbf{talk} can be passed to \textbf{announce} without a problem. 
If an object has no method \textbf{talk} (lets say: a number), the function \textbf{announce} will print the other message instead of calling our method.

\lstinputlisting[language={[KB]Python}, linerange={25-34}, style=codeStyle]{../codes/python/polymorphism/subtype.py}

Lets try an array with different elements 

\lstinputlisting[language={[KB]Python}, linerange={36-42}, style=codeStyle]{../codes/python/polymorphism/subtype.py}

\subsection{Ruby}

Lets create our classes: all of them afford a method \textbf{talk} either directly or by inheritance.

\lstinputlisting[language={[KB]Ruby}, linerange={3-17}, style=codeStyle]{../codes/ruby/polymorphism/subtype.rb}

A Ruby function does not specify the types of its parameters. 
In \textbf{announce}, it is clear that the object named \textbf{talker} must define a method called \textbf{talk}. 
So, we have to verify if the object affords this method using a function \keyword[Ruby]{method\_defined?} which is applied to classes  (we have to retrieve the object's class first).

\lstinputlisting[language={[KB]Ruby}, linerange={19-26}, style=codeStyle]{../codes/ruby/polymorphism/subtype.rb}

Then, objects created from any class affording the method \textbf{talk} can be passed to \textbf{announce} without a problem. 
If an object has no method \textbf{talk} (lets say: a number), the function \textbf{announce} will print the other message instead of calling our method.

\lstinputlisting[language={[KB]Ruby}, linerange={28-37}, style=codeStyle]{../codes/ruby/polymorphism/subtype.rb}

Lets try an array with different elements 

\lstinputlisting[language={[KB]Ruby}, linerange={39-47}, style=codeStyle]{../codes/ruby/polymorphism/subtype.rb}

\section{Type manipulation}

There exists some methods afforded by \ac{oop} languages to manipulate types. 
Even languages with dynamic type checking have some of these methods. 
For each programming language, we will see these points:
\begin{itemize}
	\item \textbf{Get object type}: How to get the type of a given object.
	\item \textbf{Is instance of}: Verify if an object is generated from a class/prototype. 
	\item \textbf{Members existence}: Check if a member (field or method) exists in a given object.
	\item \textbf{Type casts}: Downcasting is the act of changing a reference of a base class to one of its derived class.
	It is useful in case of statically typed languages like C++ and Java.
\end{itemize}

\subsection{C++}

Lets create our classes. 
A class is said to be \nameword{polymorphic} if it has at least one virtual method (defined in it or inherited).
In our example, the class \textbf{Person} has one virtual method, while the class \textbf{Machine} has none.

\lstinputlisting[language={[KB]C++}, linerange={5-24}, style=codeStyle]{../codes/cpp/polymorphism/type_manip.cpp}

C++ affords a function \keyword[C++]{typeid} from the standard library \nameword{typeinfo}. 
You can either print the type of a pointer or, dynamically, the type of the object pointed to.
A type must be polymorphic so this function can operate dynamically on it.

\lstinputlisting[language={[KB]C++}, linerange={40-46, 52-53, 57-57, 59-59}, style=codeStyle]{../codes/cpp/polymorphism/type_manip.cpp}

To verify if an object pointed to by a pointer is of a certain type dynamically, you can try to cast it to that type and verify if it succeeded.
A function proposed by \citet{2014-panzenbock} which works polymorphic pointers.

\lstinputlisting[language={[KB]C++}, linerange={26-29}, style=codeStyle]{../codes/cpp/polymorphism/type_manip.cpp}

Then, a pointer is tested as follows

\lstinputlisting[language={[KB]C++}, linerange={64-64}, style=codeStyle]{../codes/cpp/polymorphism/type_manip.cpp}

In C++, you can check if a class has a member using a mechanism called: \ac{sfinae}. 
For example, this is a template to verify if classes has a method \textbf{learn}.

\lstinputlisting[language={[KB]C++}, linerange={31-38}, style=codeStyle]{../codes/cpp/polymorphism/type_manip.cpp}

It can be used as follows 

\lstinputlisting[language={[KB]C++}, linerange={76-76}, style=codeStyle]{../codes/cpp/polymorphism/type_manip.cpp}

There are two types of down-casting (among others): \keyword[C++]{static\_cast} and \keyword[C++]{dynamic\_cast}. 
Using the first one, type check is performed during compilation. 
The dynamic cast checks types at runtime and return \keyword[C++]{NULL} in case of pointers, or throw an exception in case of references.

\lstinputlisting[language={[KB]C++}, linerange={83-87}, style=codeStyle]{../codes/cpp/polymorphism/type_manip.cpp}

\subsection{Java}

Lets define some classes

\lstinputlisting[language={[KB]Java}, linerange={66-81}, style=codeStyle]{../codes/java/src/polymorphism/TypeManip.java}

To find the class of an object, you can use the method \keyword[Java]{getClass} inherited from the universal class \keyword[Java]{Object}. 
A class \keyword[Java]{Class} affords a method \keyword[Java]{getName} which returns its name as a \keyword[Java]{String}. 

\lstinputlisting[language={[KB]Java}, linerange={5-5,23-26,32-32,63-64}, style=codeStyle]{../codes/java/src/polymorphism/TypeManip.java}

To verify if an object is an instance of a class, the keyword \keyword[Java]{instanceof} is used.

\lstinputlisting[language={[KB]Java}, linerange={41-41}, style=codeStyle]{../codes/java/src/polymorphism/TypeManip.java}

It is possible to verify if an object has a field or method in Java using reflection. 
You have to import \textbf{java.lang.reflect.*} which contains two classes \keyword[Java]{Method} and \keyword[Java]{Field}. 
A class \keyword[Java]{Class} affords two methods \keyword[Java]{getMethod} which returns a method, and \keyword[Java]{getField} which returns a field. 
These two methods when they cannot find the member in question, they throw an exception. 
To execute a \keyword[Java]{Method}, the method \keyword[Java]{invoke} is used, given the first parameter which is the object affording this method and a list of objects used as parameters. 
Likewise, to recover a value \keyword[Java]{Field}, the method \keyword[Java]{get} which returns an \keyword[Java]{Object}; or you can use other methods which return some other types if you know the type of your field. 
Reflection is not usually used since your methods, mostly, will have a defined typed parameters.

\lstinputlisting[language={[KB]Java}, linerange={6-22}, style=codeStyle]{../codes/java/src/polymorphism/TypeManip.java}

A variable with a type of some class referring an object with a type of its subclass cannot call the subclass's methods. 
If you try to do that, you will have a compiler error telling you that the object does not afford that method.
So, you have to tell the compiler that the variable is, in fact, referencing an object of the subclass. 
This is called: type casting, and more precisely: downcasting.

\lstinputlisting[language={[KB]Java}, linerange={57-62}, style=codeStyle]{../codes/java/src/polymorphism/TypeManip.java}

\subsection{Javascript}

Lets create our classes (\ac{es6} style, \ac{es5} code is afforded as well).

\lstinputlisting[language={[KB]Javascript}, linerange={3-23, 35-38}, style=codeStyle]{../codes/javascript/polymorphism/es6/type_manip.js}

Using \keyword[Javascript]{typeof}, the type you will get is \textbf{object} for our new objects. 
If you want to know the prototype used to create an object, each object has a property called \keyword[Javascript]{constructor} which refers to the class/prototype used to create the object.

\lstinputlisting[language={[KB]Javascript}, linerange={45-45}, style=codeStyle]{../codes/javascript/polymorphism/es6/type_manip.js}

To verify if an object is an instance of a class, the keyword \keyword[Javascript]{instanceof} is used.
You can realize that a prototype can inherit just one prototype, and even using \keyword[Javascript]{Object.assign} the first prototype is considered as the superclass and the others are just mixins.

\lstinputlisting[language={[KB]Javascript}, linerange={53-53}, style=codeStyle]{../codes/javascript/polymorphism/es6/type_manip.js}

To verify if an object has a method, simply check if the method is of type ``\textbf{function}". 
To verify if an object has a field (property in Javascript lingua), you can use \keyword[Javascript]{hasOwnProperty} inherited from the universal class \keyword[Javascript]{Object}.
Some solutions on the web suggest that you verify if a field's type is not ``\textbf{undefined}", but a field can exist for an object with ``undefined" value.

\lstinputlisting[language={[KB]Javascript}, linerange={25-33}, style=codeStyle]{../codes/javascript/polymorphism/es6/type_manip.js}

There is no cast on Javascript, all happens in execution time. 
If you called a non existing method of an object, you will have an error.

\subsection{Lua}

Lets create our classes: classes are simple tables in our case. 
We add a field called \textbf{\_\_NAME} to each class to afford its name.

\lstinputlisting[language={[KB]Lua}, linerange={1-41}, style=codeStyle]{../codes/lua/polymorphism/type_manip.lua}

Using the function \keyword[Lua]{type}, Lua can afford the type of a given element: nil, boolean, number, string, userdata, function, thread, and table. 
Which means, creating a table representing a class and setting it as a metatable to another does not create a new type. 
There are various methods to find the class (metatable) of an object \citep{2013-manura}, one of them is to define a field \textbf{\_\_NAME} in each class. 
This field must not be overridden in the copies (objects), otherwise it will not function appropriately.
We can define a new function called \textbf{typeOf} based on this field. 
But, if we create an object (table) dynamically with a field \textbf{\_\_NAME} equals to one of our classes, you can fool this function. 
An example of this is the object \textbf{cat} which will be considered as a \textbf{Person}.
You can go beyond the function afforded in our code by verifying if the object (table) has a metatable. 
If it does not then its type is, lets say, \textbf{Class} whatever the \textbf{\_\_NAME} is. 
If it has a metatable, you can check if its name is similar to its metable's. 
If not, you can consider it as a class; otherwise, the \textbf{\_\_NAME} will be its type.

\lstinputlisting[language={[KB]Lua}, linerange={47-51}, style=codeStyle]{../codes/lua/polymorphism/type_manip.lua}

There is no built-in function to verify if an object is instance of a class in Lua, since objects and classes are just tables.
But using the solution used to verify the types, we can define a function \textbf{instanceof} which compare \textbf{\_\_NAME} of an object with those of its metatable chains (metatable and metatable of metatable, etc.).

\lstinputlisting[language={[KB]Lua}, linerange={63-71}, style=codeStyle]{../codes/lua/polymorphism/type_manip.lua}

To verify if an object has a method, first you have to verify if the object is a table then if this method is of type \textbf{function}. 
Likewise, to verify the existence of a field, first you have to verify if the object is a table then if this field is not \keyword[Lua]{nil} (also it is not a function). 

\lstinputlisting[language={[KB]Lua}, linerange={82-96}, style=codeStyle]{../codes/lua/polymorphism/type_manip.lua}

There is no cast in Lua, all happens in execution time and you will not be needing it.

\subsection{Perl}

Lets create our classes: Perl's classes are packages and objects are hashes blessed to a package.

\lstinputlisting[language={[KB]Perl}, linerange={1-19}, style=codeStyle]{../codes/perl/polymorphism/type_manip.pl}

The function \keyword[Perl]{ref} gives the package's name to which a hash is blessed.

\lstinputlisting[language={[KB]Perl}, linerange={25-25}, style=codeStyle]{../codes/perl/polymorphism/type_manip.pl}

There is a built-in method \keyword[Perl]{isa} inherited from the universal class \keyword[Perl]{UNIVERSAL} which is used to determine the type of a reference. 
But before using it, we have to verify if the object is blessed.

\lstinputlisting[language={[KB]Perl}, linerange={33-37}, style=codeStyle]{../codes/perl/polymorphism/type_manip.pl}

To verify if an object has a method, Perl affords a method \keyword[Perl]{can} inherited from the universal class \keyword[Perl]{UNIVERSAL}.
To verify the existence of a field, a function \keyword[Perl]{exists} can be used.

\lstinputlisting[language={[KB]Perl}, linerange={49-60}, style=codeStyle]{../codes/perl/polymorphism/type_manip.pl}

There is no cast in Perl, all happens in execution time and you will not be needing it.

\subsection{PHP}

Lets create our classes and one interface

\lstinputlisting[language={[KB]PHP}, linerange={3-19}, style=codeStyle]{../codes/php/polymorphism/type_manip.php}

The function \keyword[PHP]{gettype} returns the type of a variable: the objects are of type \textbf{object}. 
To get the class of an object, the function \keyword[PHP]{get\_class} is used; it cannot be used with primitive types such as integers.

\lstinputlisting[language={[KB]PHP}, linerange={25-25,27-27}, style=codeStyle]{../codes/php/polymorphism/type_manip.php}

To verify if an object is an instance of a class (or its parents), the keyword \keyword[PHP]{instanceof} is used.
Also, there is a function \keyword[PHP]{is\_a} which serves the same purpose, but the class's name is passed as a string or a variable containing the name.

\lstinputlisting[language={[KB]PHP}, linerange={38-41}, style=codeStyle]{../codes/php/polymorphism/type_manip.php}

To verify if an object has a method, PHP affords a function \keyword[PHP]{method\_exists}.
As for fields, it affords a function \keyword[PHP]{property\_exists}.
These two functions work on objects; if you pass them another type such as an integer, an error will occur. 
To test if a variable is an object, the function \keyword[PHP]{is\_object} can be used.

\lstinputlisting[language={[KB]PHP}, linerange={47-57}, style=codeStyle]{../codes/php/polymorphism/type_manip.php}

There is no cast in PHP, all happens in execution time and you will not be needing it.

\subsection{Python}

Lets create our classes

\lstinputlisting[language={[KB]Python}, linerange={4-23}, style=codeStyle]{../codes/python/polymorphism/type_manip.py}

The function \keyword[Python]{type} returns the type of a variable. 
Because everything in Python is an object, even built-in types such as \textbf{int} accept this function.
Another mechanism to get an object's class is using the attribute \keyword[Python]{\_\_class\_\_}. 
To get the name of a class as a string, you can use the  attribute \keyword[Python]{\_\_name\_\_}. 

\lstinputlisting[language={[KB]Python}, linerange={28-29, 32-32}, style=codeStyle]{../codes/python/polymorphism/type_manip.py}

To verify if an object is an instance of a class (or its parents), the function \keyword[Python]{isinstance} is used.

\lstinputlisting[language={[KB]Python}, linerange={41-41,44-44}, style=codeStyle]{../codes/python/polymorphism/type_manip.py}

To verify if an object has a member, Python affords a function \keyword[Python]{hasattr}.
Another function \keyword[Python]{getattr} is used to get a member if it exists or returns a defined value otherwise.
To verify if a member is a method, the function \keyword[Python]{callable} can be used.

\lstinputlisting[language={[KB]Python}, linerange={50-62}, style=codeStyle]{../codes/python/polymorphism/type_manip.py}

There is no cast in Python, all happens in execution time and you will not be needing it.

\subsection{Ruby}

Lets create our classes and a mixin 

\lstinputlisting[language={[KB]Ruby}, linerange={3-28}, style=codeStyle]{../codes/ruby/polymorphism/type_manip.rb}

To find the class of an object, you can simply call the method \keyword[Ruby]{class}.
Then, the method \keyword[Ruby]{to\_s} is used to get the sting representation. 
In Ruby, everything is an object and even numbers has a class.

\lstinputlisting[language={[KB]Ruby}, linerange={34-34, 37-37}, style=codeStyle]{../codes/ruby/polymorphism/type_manip.rb}

To verify if an object is an instance of a class (or its parents), the functions \keyword[Ruby]{kind\_of?} and \keyword[Ruby]{is\_a?} can be used. 
The two methods are equivalent; they verify if an object inherit from a mixin as well.
There is another method \keyword[Ruby]{instance\_of?} which verify if an object was instantiated from a specific class (it does not verify its parents).

\lstinputlisting[language={[KB]Ruby}, linerange={43-43,48-49,54-54}, style=codeStyle]{../codes/ruby/polymorphism/type_manip.rb}

To verify if an object has a method, a class of an object has a method \keyword[Ruby]{method\_defined?}.
As for fields, they are encapsulated (protected mode) and cannot be accessed outside the class's hierarchy unless you define accessors (getters and setters). 
The accessors are methods and can be checked the same way a normal method is checked. 
In our example, \textbf{Professor} class has a field \textbf{nbr} and a reader accessor (getter). 
To verify the setter (not shown in the example), you write  \textbf{obj.method\_defined? :nbr=}.

\lstinputlisting[language={[KB]Ruby}, linerange={60-68}, style=codeStyle]{../codes/ruby/polymorphism/type_manip.rb}

There is no cast in Ruby because variable type is dynamic.

\section{Methods overloading}

It is the feature of a class having multiple methods defined by the same identifier and different parameters. 
Statically-typed programming languages are, usually, the ones affording methods overloading since they enforce type checking during function calls.
In this section, we will verify these properties:
\begin{itemize}
	\item If the programming language supports methods overloading. 
	If not, how to afford something similar. 
	\item If we can overload methods over inheritance
\end{itemize}

\subsection{C++}

In C++, overloading is permitted in the same class 

\lstinputlisting[language={[KB]C++}, linerange={4-8,16-18,22-23,31-32}, style=codeStyle]{../codes/cpp/polymorphism/overloading.cpp}

When the same method is overloaded in a subclass, the new definition will hide those of the parent class;
if they are called, this will generate a compilation error. 
To call them, there are two ways: import them using \keyword[C++]{using} declaration, or by calling them explicitly \citep{2009-bailey}.
\lstinputlisting[language={[KB]C++}, linerange={10-17,19-19,26-32}, style=codeStyle]{../codes/cpp/polymorphism/overloading.cpp}

\subsection{Java}

In Java, overloading is permitted in the same class or its subclasses as well.

\lstinputlisting[language={[KB]Java}, linerange={3-6,9-10,13-26}, style=codeStyle]{../codes/java/src/polymorphism/Overloading.java}

\subsection{Javascript}

Javascript does not support method overloading. 
A similar thing can be achieved by testing arguments types.

\lstinputlisting[language={[KB]Javascript}, linerange={3-15,31-33}, style=codeStyle]{../codes/javascript/polymorphism/es6/overloading.js}

If there are many arguments, you can use the keyword \keyword[Javascript]{arguments} which is an Array-like object accessible inside functions that contains the values of the arguments passed to that function.
Subclasses can define the same method, verify the arguments for new definitions and delegate the control to the super's methods.

\lstinputlisting[language={[KB]Javascript}, linerange={17-28,36-40}, style=codeStyle]{../codes/javascript/polymorphism/es6/overloading.js}

\subsection{Lua}

Lua does not support method overloading. 
A similar behavior can be achieved by using a \nameword{variadic function} and testing arguments number and types.
You can select a parameter using the function \keyword[Lua]{select} which returns the nth parameter and its following parameters \citep{2011-berteig}.
The same function can be used to know te number of parameters.

\lstinputlisting[language={[KB]Lua}, linerange={2-20,37-39}, style=codeStyle]{../codes/lua/polymorphism/overloading.lua}

Subclasses can define the same method, verify the arguments for new definitions and delegate the control to the super's methods.
If you do not want to use \keyword[Lua]{select}, you can put the arguments in a list and access the parameters as list members.

\lstinputlisting[language={[KB]Lua}, linerange={22-32,40-43}, style=codeStyle]{../codes/lua/polymorphism/overloading.lua}

\subsection{Perl}

Perl does not support method overloading. 
A subroutine receive its parameters using an array \keyword[Perl]{@\_}; thus, it is possible to verify the number of arguments (the first one is the object). 
If you pass more than the acceptable amount of arguments, this will not be a problem (as shown in the example).
Perl does not make a difference between scalar types (eg. numbers and strings), so you can just verify blessed objects using \keyword[Perl]{ref}. 

\lstinputlisting[language={[KB]Perl}, linerange={1-11,28-31}, style=codeStyle]{../codes/perl/polymorphism/overloading.pl}

Subclasses can define the same method, verify the arguments for new definitions and delegate the control to the super's methods.

\lstinputlisting[language={[KB]Perl}, linerange={13-25,34-37}, style=codeStyle]{../codes/perl/polymorphism/overloading.pl}

You can check out \nameword{MooseX::MultiMethods}\footnote{MooseX::MultiMethods: \url{https://metacpan.org/pod/MooseX::MultiMethods}} for Multi Method Dispatch based on Moose type constraints.

\subsection{PHP}

PHP does not support \ac{oop}'s method overloading; Overloading in PHP is entirely a different concept (check PHP documentation). 
However, there is a mechanism to achieve a similar thing using a \nameword{variadic function}: the number of arguments is recovered using a function \keyword[PHP]{func\_num\_args}, and the arguments are recovered using a function \keyword[PHP]{func\_get\_arg} \citep{2011-tan}. 

\lstinputlisting[language={[KB]PHP}, linerange={3-18, 30-32}, style=codeStyle]{../codes/php/polymorphism/overloading.php}

Subclasses can define the same method, verify the arguments for new definitions and delegate the control to the super's methods.
Arguments can, also, be passed as a list as follows. 

\lstinputlisting[language={[KB]PHP}, linerange={20-27,35-38}, style=codeStyle]{../codes/php/polymorphism/overloading.php}

\subsection{Python}

Python does not support method overloading. 
Using default values and type checking, you can achieve a similar behavior. 

\lstinputlisting[language={[KB]Python}, linerange={4-10,21-23}, style=codeStyle]{../codes/python/polymorphism/overloading.py}

You can, also, use a \nameword{variadic function} for unknown amount of parameters.
Subclasses can define the same method, verify the arguments for new definitions and delegate the control to the super's methods.

\lstinputlisting[language={[KB]Python}, linerange={11-18,26-29}, style=codeStyle]{../codes/python/polymorphism/overloading.py}

You can define an \ac{api} for methods overloading such as \nameword{pythonlangutil}\footnote{pythonlangutil: \url{https://github.com/ehsan-keshavarzian/pythonlangutil}}.

\subsection{Ruby}

Ruby does not support method overloading. 
Using default values and type checking, you can achieve a similar behavior. 

\lstinputlisting[language={[KB]Ruby}, linerange={3-14,29-31}, style=codeStyle]{../codes/ruby/polymorphism/overloading.rb}

You can, also, use a \nameword{variadic function} for unknown amount of parameters.
Subclasses can define the same method, verify the arguments for new definitions and delegate the control to the super's methods.

\lstinputlisting[language={[KB]Ruby}, linerange={15-27,34-37}, style=codeStyle]{../codes/ruby/polymorphism/overloading.rb}


\section{Methods overriding}

Method overriding allows a subclass to provide a specific implementation of an existing parent's method. 
The method must have the same signature (name and arguments) and return type (if the language allows it) as the parent's. 
Some methods can be abstract or final: abstract methods must be overridden and final methods cannot be overridden (see inheritance chapter).

\subsection{C++}

To override a method, you have to define a subclass with the same method's signature and return type. 
A method to be overridden can be defined using the keyword \keyword[C++]{virtual} or not. 
A virtual method is resolved at run-time using a mechanism called \nameword{Late binding}. 
While a normal method is resolved at compilation-time using a mechanism called \nameword{Early (Static) binding}.

\lstinputlisting[language={[KB]C++}, linerange={4-14}, style=codeStyle]{../codes/cpp/polymorphism/overriding.cpp}

Early binding follows the pointer's type when the method is called; therefore, if the pointer's type is a superclass, the implementation which will be used is that of the superclass are not the object's. 
In the contrary, a method defined using \keyword[C++]{virtual} (Late binding) will be called based on the object's real type and not that of the pointer. 

\lstinputlisting[language={[KB]C++}, linerange={16-31}, style=codeStyle]{../codes/cpp/polymorphism/overriding.cpp}

\subsection{Java}

In Java, you can override a method using the same signature and return type. 
As mentioned in Encapsulation chapter, the visibility mode can be set to more permissive one (for example, from private to public).
To be sure you overrode an existing method, you can use the annotation \keyword[Java]{@Override} which tells the compiler to check if the method actually exists in one of the superclasses.

\lstinputlisting[language={[KB]Java}, linerange={12-19}, style=codeStyle]{../codes/java/src/polymorphism/Overriding.java}

An object of a class referred by a reference of its superclass will execute its specific implementation and not that of the reference. 

\lstinputlisting[language={[KB]Java}, linerange={3-10}, style=codeStyle]{../codes/java/src/polymorphism/Overriding.java}

\subsection{Javascript}

In Javascript, you can override a method by just defining it using the same signature. 
The object of the subclass will use the new defined method.

\lstinputlisting[language={[KB]Javascript}, linerange={3-14}, style=codeStyle]{../codes/javascript/polymorphism/es6/overriding.js}

\subsection{Lua}

In Lua, you can override a method by just defining it using the same signature. 
The object of the subclass will use the new defined method.

\lstinputlisting[language={[KB]Lua}, linerange={1-27}, style=codeStyle]{../codes/lua/polymorphism/overriding.lua}

\subsection{Perl}

In Perl, you can override a method by just defining it using the same name. 
The object of the subclass will use the new defined method.

\lstinputlisting[language={[KB]Perl}, linerange={1-14}, style=codeStyle]{../codes/perl/polymorphism/overriding.pl}

\subsection{PHP}

In PHP, you can override a method by just defining it using the same signature. 
The object of the subclass will use the new defined method.

\lstinputlisting[language={[KB]PHP}, linerange={3-14}, style=codeStyle]{../codes/php/polymorphism/overriding.php}

\subsection{Python}

In Python, you can override a method by just defining it using the same signature. 
The object of the subclass will use the new defined method.

\lstinputlisting[language={[KB]Python}, linerange={4-15}, style=codeStyle]{../codes/python/polymorphism/overriding.py}

\subsection{Ruby}

In Ruby, you can override a method by just defining it using the same signature. 
The object of the subclass will use the new defined method.

\lstinputlisting[language={[KB]Ruby}, linerange={3-18}, style=codeStyle]{../codes/ruby/polymorphism/overriding.rb}

\begin{discussion}
	
	Polymorphism allows us to process objects differently depending on their data type or class. 
	Based on type compatibility and equivalence, three type systems can be determined:
	\begin{itemize}
		\item \textbf{Nominal type system}: The compatibility and equivalence of data types is is determined by explicit declarations or name.
		Example: C++, C\#, Java.
		\item \textbf{Structural type system}: The compatibility and equivalence of data types is is determined by their structures or definitions: if they have the same definition then they are equivalent. 
		Example: OCaml, Scala, Go.
		\item \textbf{Duck typing}: The compatibility is determined by the presence of certain methods and fields.
		``\textit{If it walks like a duck and talks like a duck, it must be a duck}".
		Example: Javascript, Python, Ruby.
	\end{itemize}
	
	Table~\ref{tab-polymorphism} represents a comparison between some OOP languages based on their polymorphism capacities: 
	\begin{itemize}
		\item \textbf{Type compatibility}: Nominal, structural or duck.
		\item \textbf{Get object type}: How to get the type of a given object.
		\item \textbf{Is instance of}: Verify if an object is generated from a class/prototype. 
		\item \textbf{Members existence}: Check if a member (field or method) exists in a given object.
		\item \textbf{Type casts}: Downcasting is the act of changing a reference of a base class to one of its derived class.
		\item \textbf{Overloading}: Having methods overloading or not.
	\end{itemize}
	As for overriding, no need to compare since all of our languages afford this property.
	
	\begin{landscape}
		\extrarowsep = 0pt 
		
		%	\doublerawsep = 1.5pt
		\begin{longtabu} to \linewidth %
			{
				X[1,p]|[5pt white]
				X[0.75,p]|[5pt white]
				X[2,p]|[5pt white]
				X[2,p]|[5pt white]
				X[2,p]|[5pt white]
				X[2,p]|[5pt white]
				X[0.5,p]|[5pt white]
			} %{llllllll} 
			%	\begin{longtabu} {p{1cm}p{1cm}p{5cm}p{1cm}p{1cm}p{2cm}p{2cm}p{1cm}}
			\caption{Polymorphism comparison}%
			\label{tab-polymorphism}\\
			
			%		\hline\hline
			\rowcolor{indigo}
			\rowfont{\bfseries\color{white}}
			{Language} &
			{Type compatibility} &
			{Object's type} &
			{Instance of} &
			{members existence} & 
			{cast} &
			{overloading}\\
			%		\hline\hline
			&&&&&&\\
			\endfirsthead
			
			%		\hline\hline
			\rowcolor{indigo}
			\rowfont{\bfseries\color{white}}
			{Language} &
			{Type system} &
			{Object's type} &
			{Instance of} &
			{members existence} & 
			{cast} &
			{overloading}\\
			%		\hline\hline
			&&&&&&\\
			\endhead
			
			\taburowcolors{indigo!20!white .. black!10!white}
			
			{\bfseries\color{indigo}C++} & %Language
			Nominal& %type system
			typeid(obj)& %object's type
			Using cast& %Instance of
			/ & %members existance
			(Cls) obj; \newline 
			static\_cast<Cls*> (ptr) \newline 
			dynamic\_cast<Cls*> (ptr) & %cast
			Yes\\%overloading
			%		\hline
			
			{\bfseries\color{indigo}Java} & %Language
			Nominal& %Type system
			obj.getClass()& %object's type
			obj instanceof Cls& %Instance of 
			using reflection & %member existance
			(Cls) obj & %cast
			Yes\\%overloading
			%		\hline
			
			{\bfseries\color{indigo}Javascript} & %Language
			Duck& %Type system
			Type: typeof obj \newline Class: obj.constructor.name& %object's type
			obj instanceof Cls& %instance of
			obj.hasOwnProperty ("property")& %member existance
			/& %cast
			/\\%overloading
			%		\hline
			
			{\bfseries\color{indigo}Lua} & %Language
			Duck& %Type system
			Type: type(obj) \newline Class: /& %object's type
			/& %Instance of
			access and verify type& %Member existance
			/& %cast
			/\\%overloading
			%		\hline
			
			{\bfseries\color{indigo}Perl} & %Language
			& %Type system
			ref(obj)& %Object's type 
			obj->isa(Cls)& % instance of
			obj->can("method")& %member existance
			/& % cast
			/\\%overloading
			%		\hline
			
			{\bfseries\color{indigo}PHP} & %Language
			Duck?& %Type system
			Type: gettype(obj) \newline Class: get\_class(obj) & %object's type
			obj instanceof Cls \newline is\_a(obj, "Cls") & % Instance of
			method\_exists(obj, "method") \newline property\_exists(obj, "field") & % member existance
			/ & %cast
			/ \\%overloading
					\hline
			
			{\bfseries\color{indigo}Python} & %Language
			Duck& %Type system
			type(obj) \newline obj.\_\_class\_\_& %Object's type
			isinstance(obj, Cls)& % instance of
			hasattr(obj, "property")& %member existance
			/& %cast
			/\\%oveloading
			%		\hline
			
			{\bfseries\color{indigo}Ruby} & %Language
			Duck& %Type system
			obj.class& %Object's type
			obj.is\_a?(Cls) \newline instance\_of?(Cls) \newline kind\_of?(Cls)& %instance of
			obj.class.method\_defined? :method & %member existance 
			/& %cast
			/\\%overloading
			%		\hline
			
			%		\hline\hline
		\end{longtabu}
	\end{landscape}
	
\end{discussion}

%=====================================================================
\ifx\wholebook\relax\else
% \cleardoublepage
% \bibliographystyle{../use/ESIbib}
% \bibliography{../bib/RATstat}
	\end{document}
\fi
%=====================================================================