%=====================================================================
\ifx\wholebook\relax\else
	\documentclass[12pt]{book}
	\usepackage{amsmath,amssymb}             % AMS Math
\usepackage[utf8]{inputenc}
%\usepackage[T1]{fontenc}

%\usepackage[pdftex]{graphicx}

%\usepackage{listingsutf8}
%\usepackage{xcolor}
%\usepackage{times}
\usepackage{array}
\usepackage{natbib}
\usepackage{lscape}%to flip tables in a page
\usepackage{pdflscape}
\usepackage{longtable}
\usepackage{tabu}
\usepackage{wrapfig}

%\usepackage[english]{babel}

\bibliographystyle{engdnat}%unsrtnat, plainnat

%\usepackage{pgf-umlcd}




\hypersetup{
	colorlinks,
	urlcolor=blue
}

\renewcommand{\UrlFont}{\ttfamily\footnotesize}

\DeclareAcronym{oop}{
	short = OOP ,
	long  = Object-oriented programming ,
	class = abbrev
}

\DeclareAcronym{oo}{
	short = OO ,
	long  = Object-oriented ,
	class = abbrev
}

\DeclareAcronym{ecma}{
	short = ECMA ,
	long  = European Computer Manufacturers Association ,
	class = abbrev
}

\DeclareAcronym{cli}{
	short = CLI ,
	long  = Command-Line Interface ,
	class = abbrev
}

\DeclareAcronym{cgi}{
	short = CGI ,
	long  = Common Gateway Interface ,
	class = abbrev
}

%\makeglossaries

%\newacronym{oop}{OOP}{Object-oriented programming}
	\begin{document}
\fi
%=====================================================================

% ===================================================
%                 COPYRIGHT
% ===================================================
\chapter*{\textarab{حقوق النشر}}
\addcontentsline{toc}{section}{\textarab{حقوق النشر}}

\begin{center}
	{\Huge \textbf{Object-oriented programming for coders}}
\end{center}


\noindent
2016, 2018  \textcopyright\ Abdelkrime Aries <\href{mailto://kariminfo0@gmail.com}{kariminfo0@gmail.com}> \\[0.5cm]
This book is provided under the Creative Commons Attribution-Share Alike 4 License Agreement (\url{https://creativecommons.org/licenses/by-sa/4.0/}).

\vspace{0.5cm}
\noindent
Project URL: \url{https://github.com/kariminf/oop4coders}

\vspace{1cm}

\scriptsize %\tiny
\vspace{1cm}
\noindent
You are free to:
\begin{itemize}
\item \textbf{Share -} copy and redistribute the material in any medium or format
\item \textbf{Adapt -} remix, transform, and build upon the material for any purpose, even commercially.
\end{itemize}
The licensor cannot revoke these freedoms as long as you follow the license terms.

\vspace{1cm}
\noindent
Under the following terms:\\
\begin{tabular}{m{.1\textwidth}m{.8\textwidth}}
%
& %\includegraphics[width=1cm]{img/by.png} & 
\textbf{Attribution —} You must give appropriate credit, provide a link to the license, and indicate if changes were made. You may do so in any reasonable manner, but not in any way that suggests the licensor endorses you or your use. \\
%
& %\includegraphics[width=1cm]{img/sa.png} &
\textbf{ShareAlike —} If you remix, transform, or build upon the material, you must distribute your contributions under the same license as the original.\\
%
&
\textbf{No additional restrictions —} You may not apply legal terms or technological measures that legally restrict others from doing anything the license permits.
%
\end{tabular}

\vspace{.5cm}
\noindent
With the understanding that:
\begin{description}
\item[] \textbf{Waiver —} Any of the above conditions can be \textbf{waived} if you get permission from the copyright holder.

\item[] \textbf{Public Domain —} Where the work or any of its elements is in the \textbf{public domain} under applicable law, that status is in no way affected by the license.

\item[] \textbf{Other Rights —} In no way are any of the following rights affected by the license:
\begin{itemize}
\item Your fair dealing or \textbf{fair use} rights, or other applicable copyright exceptions and limitations;
\item The author's \textbf{moral} rights;
\item Rights other persons may have either in the work itself or in how the work is used, such as \textbf{publicity} or privacy rights.
\end{itemize}
\end{description}
\normalsize

% ===================================================
%                   CREDITS
% ===================================================
\chapter*{\textarab{شكر}}
\addcontentsline{toc}{section}{\textarab{شكر}}

Information retrieval:
\begin{itemize}
	\item Google search engine
	\item Wikipedia
\end{itemize}

Logos: 
\begin{itemize}
	\item C++ Logo: Jeremy Kratz, \url{https://commons.wikimedia.org/wiki/File:ISO_C\%2B\%2B_Logo.svg}
	\item Java Duke Logo: sbmehta converted, \url{https://commons.wikimedia.org/wiki/File:Duke_(Java_mascot)_waving.svg}
	\item Javascript unofficial Logo: Chris Williams,  \url{https://commons.wikimedia.org/wiki/File:Unofficial_JavaScript_logo_2.svg}
	\item Lua project Logo: Lua.org, \url{https://commons.wikimedia.org/wiki/File:Lua-logo-nolabel.svg}
	\item Perl Logo: cannot use both since there are restrictions on commercial use, while this book allows commercial use.
	%https://www.perlfoundation.org/trademarks.html
	\item PHP Logo: Colin Viebrock, \url{https://commons.wikimedia.org/wiki/File:PHP-logo.svg}
	\item Python Logo: Python community, \url{https://www.python.org/community/logos/}
	\item Ruby Logo: Yukihiro Matsumoto, \url{https://commons.wikimedia.org/wiki/File:Ruby_logo.svg}
\end{itemize}

Cover:
\begin{itemize}
	\item Gimp: Image Manipulation Program
	\item Mountain gorilla (Gorilla beringei beringei) eating, by Charles J Sharp, License: CC-By-SA-4.0, Link: \url{https://commons.wikimedia.org/wiki/File:Mountain_gorilla_(Gorilla_beringei_beringei)_eating.jpg}
	\item Rain in Tena, by Lion Hirth, License: ppublic domain, Link: \url{https://commons.wikimedia.org/wiki/File:Rain_in_Tena.jpg}.
\end{itemize}

Fonts: 
\begin{itemize}
	\item Google fonts 
\end{itemize}

Editing:
\begin{itemize}
	\item \LaTeX: a document preparation system
	\item Atom: Source code editor.
	\item TeXstudio: Text editor for Latex
	\item KDE Neon: Linux system based on Ubuntu
\end{itemize}

Hosting and version control:
\begin{itemize}
	\item Git: a version-control system
	\item Github: a web-based hosting service for version control using Git.
\end{itemize}

And, of course, all the readers.

% ===================================================
%                 VERSIONS
% ===================================================
\newpage
\chapter*{\textarab{الإصدارات}}
\addcontentsline{toc}{section}{\textarab{الإصدارات}}

\noindent
\begin{itemize}
	\item August 1st, 2018: \hspace{2cm} Draft Edition 1 (version 0.1)
	\item August 15th, 2018: \hspace{2cm} Draft Edition 2 (version 0.2)
	\item September 9th, 2018: \hspace{2cm} Draft Edition 3 (version 0.3)
	\item October 13th, 2018: \hspace{2cm} First Edition (version 1.0)
\end{itemize}



%=====================================================================
\ifx\wholebook\relax\else
% \cleardoublepage
% \bibliographystyle{../use/ESIbib}
% \bibliography{../bib/RATstat}
	\end{document}
\fi
%=====================================================================