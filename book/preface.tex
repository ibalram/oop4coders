%=====================================================================
\ifx\wholebook\relax\else
	\documentclass[12pt]{book}
	\usepackage{amsmath,amssymb}             % AMS Math
\usepackage[utf8]{inputenc}
%\usepackage[T1]{fontenc}
\usepackage{hyperref}
%\usepackage[pdftex]{graphicx}
\usepackage{listings}
%\usepackage{listingsutf8}
%\usepackage{xcolor}
%\usepackage{times}
\usepackage{array}
%\usepackage{natbib}
%\usepackage[left=2.8cm,right=2.2cm,top=2.8cm,bottom=2.8cm,includefoot,includehead,headheight=13.6pt]{geometry}

%\bibliographystyle{ACM-Reference-Format-Journals}


\lstdefinestyle{codeStyle}{
  belowcaptionskip=1\baselineskip,
  breaklines=true,
  frame=L,
  xleftmargin=1.5cm,%\parindent,
  showstringspaces=false,
  basicstyle=\scriptsize\ttfamily\bfseries,
  keywordstyle=\bfseries\color{green!40!black},
  commentstyle=\itshape\color{purple!40!black},
  identifierstyle=\color{blue},
  stringstyle=\color{orange},
  backgroundcolor=\color{blue!10!white},
  lineskip=.2em,
  numbers=left
%  frameround=tttt
}

\lstdefinestyle{shellStyle}{
	belowcaptionskip=1\baselineskip,
	breaklines=true,
	frame=shadwbox,
	rulesepcolor=\color{blue},
	xleftmargin=1.25cm,
	basicstyle=\scriptsize\ttfamily\bfseries\color{green},
	backgroundcolor=\color{black},
	lineskip=.2em,
%	morekeywords={sudo},keywordstyle=\color{red},
	%  frameround=tttt
}



\hypersetup{
	colorlinks,
	urlcolor=blue
}

\renewcommand{\UrlFont}{\ttfamily\footnotesize}

	\begin{document}
\fi
%=====================================================================

\chapter*{Preface}
\addcontentsline{toc}{chapter}{Preface}

{
\merienda

There are a lot of books which talk about OOP philosophy.
So, if you want to go deeper into OOP paradigm, this isn't your place.
In this book, I want to focus on OOP from developers (programmer's) point of view.
The different implementations of OOP concepts in most object-oriented programming languages.
To this end, I will choose some programming languages, the free ones, to express that.

%Though, this book is intended for programmers to learn how to link the OOP concepts with their chosen language, it gives a solution for lazy ones.
%If you don't like programming so much, you can use Umbrello UML Modeller which can generate code for various languages.

}
\vfill
\begin{flushright}
	\LARGE\bfseries
So, ``Less talking, more coding"
\end{flushright}

\newpage

\section*{Why should you be interested?}

I presented the idea of the book on Facebook, Twitter, Google+ and Linkedin to see if it is better to continue this or let it go.
``What will be the difference between your book and the huge number of already written books on this matter?", my friend Adnan wrote on facebook.
So, let me answer this important question:
\begin{enumerate}
\item It is intended to be FREE (as gratis), so if you don't like it you won't loose money for it.
If you think you lost time, think of me and future collaborators (if any) wasting our times.
This is my advice: if you find it boring, don't continue reading.
You still can use it as a manual though, when you want to check how a specific language handles an object-oriented concept.

\item It focuses on programming languages and their styles of representing the OOP concepts.
For instance, Javascript has a very strange way to express those concepts.
Maybe, Javascript programmers won't find it strange, but as a Java programmer I find it rather interesting.

\item It is helpful when we want to compare the differences between the object oriented languages.

\item It is open source.
That is, anyone can help to enrich this book.
You can translate, correct, make your own copy from it as long as you mention the contributors and license it under CC-BY-SA 4.0.

\end{enumerate}
\vfill
\begin{flushright}
Abdelkrime Aries, March 27, 2016
\end{flushright}

%=====================================================================
\ifx\wholebook\relax\else
% \cleardoublepage
% \bibliographystyle{../use/ESIbib}
% \bibliography{../bib/RATstat}
	\end{document}
\fi
%=====================================================================
