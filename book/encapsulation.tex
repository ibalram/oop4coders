%=====================================================================
\ifx\wholebook\relax\else
	\documentclass{KodeBook}
	\usepackage{amsmath,amssymb}             % AMS Math
\usepackage[utf8]{inputenc}
%\usepackage[T1]{fontenc}
\usepackage{hyperref}
%\usepackage[pdftex]{graphicx}
\usepackage{listings}
%\usepackage{listingsutf8}
%\usepackage{xcolor}
%\usepackage{times}
\usepackage{array}
%\usepackage{natbib}
%\usepackage[left=2.8cm,right=2.2cm,top=2.8cm,bottom=2.8cm,includefoot,includehead,headheight=13.6pt]{geometry}

%\bibliographystyle{ACM-Reference-Format-Journals}


\lstdefinestyle{codeStyle}{
  belowcaptionskip=1\baselineskip,
  breaklines=true,
  frame=L,
  xleftmargin=1.5cm,%\parindent,
  showstringspaces=false,
  basicstyle=\scriptsize\ttfamily\bfseries,
  keywordstyle=\bfseries\color{green!40!black},
  commentstyle=\itshape\color{purple!40!black},
  identifierstyle=\color{blue},
  stringstyle=\color{orange},
  backgroundcolor=\color{blue!10!white},
  lineskip=.2em,
  numbers=left
%  frameround=tttt
}

\lstdefinestyle{shellStyle}{
	belowcaptionskip=1\baselineskip,
	breaklines=true,
	frame=shadwbox,
	rulesepcolor=\color{blue},
	xleftmargin=1.25cm,
	basicstyle=\scriptsize\ttfamily\bfseries\color{green},
	backgroundcolor=\color{black},
	lineskip=.2em,
%	morekeywords={sudo},keywordstyle=\color{red},
	%  frameround=tttt
}



\hypersetup{
	colorlinks,
	urlcolor=blue
}

\renewcommand{\UrlFont}{\ttfamily\footnotesize}

	\begin{document}
\fi
%=====================================================================

\chapter{Encapsulation}

\begin{introduction}
	Encapsulation is an important concept of OOP. 
	It is the mechanism of restricting direct access to some members of a class.
	It helps managing complexity when debugging source code. 
	If a field is accessed everywhere in your source code, this makes it difficult to find errors related to this field. 
	Also, if a field of a class (A) is used by a class (B) then we changed it (either its name or how it gets assigned), we have to change all the code where it appears in (B).
	In this chapter, we will show different visibility modes and how they are implemented in each language.
	You will see that there are two views: ``Many programmers are irresponsible, dim-witted, or both." and ``Programmers are responsible adults and capable of good judgment".
	So, a programming language may fall into one or another.
\end{introduction}
Some languages have just public fields, and it is up to programmers to decide if they want to access it directly. 

\section{Public members}

Public visibility mode is used to access a member of a class from everywhere. 
Public methods of a class provide an interface that allows them to be called so the object can afford some behavior. 
As for fields, public visibility mode is not preferable since it breaks encapsulation, and it has limited uses:
\begin{itemize}
	\item Constant fields can be exposed to the outside classes.%Joshua Bloch writes in Effective Java
	\item Classes which represent just data, and unlikely to be changed in the future.
\end{itemize}

\subsection{C++}

Public members are defined inside the class header using the modifier \keyword{public}. 
Every member coming after that modifier is considered as public.
If you use \keyword{struct} instead of \keyword{class}, all members are public by default.

\lstinputlisting[language=C++, linerange={6-7,12-15,25-25}, style=codeStyle]{../codes/cpp/encapsulation/person.h}

These members can be accessed anywhere

\lstinputlisting[language=C++, linerange={10-12, 14-14, 17-18,27-28}, style=codeStyle]{../codes/cpp/encapsulation/app.cpp}

An important remark: when you compile multiple files using g++, you have to include all cpp files
\begin{lstlisting}[style=shellStyle]
$ g++ app.cpp person.cpp student.cpp frnd.h
\end{lstlisting}

If you include a header twice or more, you may have an error telling you that you defined the class more than once. 
To correct this, you have to use \nameword{\#include guard}.

\lstinputlisting[language=C++, linerange={1-3,6-7,25-26}, style=codeStyle]{../codes/cpp/encapsulation/person.h}

\subsection{Java}

Every public member in Java must preceded by the keyword \keyword{public}. 

\lstinputlisting[language=Java, linerange={1-5,12-12,18-20,27-27,34-34}, style=codeStyle]{../codes/java/src/encapsulation/core/Person.java}

These members can be accessed anywhere
\lstinputlisting[language=Java, linerange={1-8,10-10,14-14,16-18}, style=codeStyle]{../codes/java/src/encapsulation/main/App.java}

\subsection{Javascript} 

All fields with the keyword \keyword{this} are public. 
All methods defined using the same keyword inside the constructor or using the prototype are public.

\lstinputlisting[linerange={3-4,12-12,16-16,22-22}, style=codeStyle]{../codes/javascript/encapsulation/person.js}

Which can be accessed anywhere

\lstinputlisting[linerange={1-1,4-4,7-7,12-12}, style=codeStyle]{../codes/javascript/encapsulation/app.js}

Actually, objects in Javascript can be updated by assigning them new fields and methods dynamically. 
This gives you the power to upgrade existing prototypes (classes) and decorate objects without the need to create more classes.
Also, as \nameword{uncle Ben} of \nameword{Spider-man} once said: ``\textit{With great power comes great responsibility}".

\lstinputlisting[linerange={1-17}, style=codeStyle]{../codes/javascript/encapsulation/decorate.js}

Here, we created two objects, but with different fields. 
Then, we can create a function which print different fields of a Person. 
Just a note: you have to verify that the member is not a function, since functions in Javascript are first class objects. 
Which means: they are a type (\keyword{function}), they are instances of \keyword{Object} and they can be treated as any variable.

\lstinputlisting[linerange={19-29}, style=codeStyle]{../codes/javascript/encapsulation/decorate.js}

\subsection{Lua}

Using tables to create objects, all members are accessible outside the class, hence public.

\lstinputlisting[language={[5.2]Lua}, linerange={1-5,7-7,11-12,14-14,17-19}, style=codeStyle]{../codes/lua/encapsulation/person.lua}

You can access them anywhere

\lstinputlisting[language={[5.2]Lua}, linerange={1-6}, style=codeStyle]{../codes/lua/encapsulation/app.lua}

\subsection{Perl}

\begin{kodequote}{Larry Wall}
	Perl doesn't have an infatuation with enforced privacy. 
	It would prefer that you stayed out of its living room because you weren't invited, not because it has a shotgun.
\end{kodequote}

I guess you get the idea: all members are public.
But, in Perl, there is always a hack to limit visibility. 
Let's see the normal case where every field and method are public. 

\lstinputlisting[language=Perl, linerange={1-18}, style=codeStyle]{../codes/perl/encapsulation/public.pl}

They can be accessed outside the class

\lstinputlisting[language=Perl, linerange={20-22}, style=codeStyle]{../codes/perl/encapsulation/public.pl}

\subsection{PHP}

The default visibility mode in PHP is public.
A function defined without a visibility modifier or a variable defined using \keyword{var} are public by default. 
It is a good practice to prefix class members with a visibility modifier, since the inverse is just used to keep compatibility with versions before 5.1.3.

\lstinputlisting[language=PHP, linerange={2-5,11-11,15-15,17-17,24-24,31-31}, style=codeStyle]{../codes/php/encapsulation/person.php}

The member with public visibility mode can be accessed anywhere.

\lstinputlisting[language=PHP, linerange={3-3,6-6,8-9,17-17}, style=codeStyle]{../codes/php/encapsulation/app.php}

\subsection{Python}

All class members are public by default; it is the only visibility mode in Python. 

\lstinputlisting[language=Python, linerange={4-4, 8-8, 12-12,15-15,19-19}, style=codeStyle]{../codes/python/encapsulation/person.py}

The members can be accessed anywhere
\lstinputlisting[language=Python, linerange={4-7, 9-9, 13-13}, style=codeStyle]{../codes/python/encapsulation/app.py}

\subsection{Ruby}

By default methods are public except for \keyword{initialize} method and the global methods defined under the object class which are private always. 
If desired, you can use the keyword \keyword{public} in a line and all methods defined after it are public. 
Fields are not public unless they are constants. 
To access them like public in other languages, you have to define accessor methods (the fields have to be properties).
In Ruby, public getters and setters are called \nameword{attribute readers} and \nameword{attribute writers}.

\lstinputlisting[language=Ruby, linerange={1-1,5-5,9-9,13-13,15-15,17-17,22-22,34-35,39-39,46-46}, style=codeStyle]{../codes/ruby/encapsulation/person.rb}

The field can be accessed for read and write. 
Actually, the field is accessed via a getter and a setter which are public by default (because they are methods).

\lstinputlisting[language=Ruby, linerange={6-9,14-14,20-20}, style=codeStyle]{../codes/ruby/encapsulation/app.rb}


\section{Protected members}

Protected visibility mode is used to access a member of a class from the class itself and its subclasses. 
For fields as for methods, it is used to allow subclasses accessing them directly. 
If you define a member as protected, keep in mind that changing it later may break subclasses.

\subsection{C++}

Protected members are defined inside the class header using the modifier \keyword{protected}. 
Every member coming after that modifier is considered as protected.

\lstinputlisting[language=C++, linerange={6-7,18-19,25-25}, style=codeStyle]{../codes/cpp/encapsulation/person.h}

These members can be accessed only from the class itself or its subclasses

\lstinputlisting[language=C++, linerange={1-5,8-8}, style=codeStyle]{../codes/cpp/encapsulation/student.cpp}

They cannot be accessed elsewhere

\lstinputlisting[language=C++, linerange={10-12,16-16,27-28}, style=codeStyle]{../codes/cpp/encapsulation/app.cpp}

\subsection{Java}

Protected members are prefixed each by the keyword \keyword{protected}. 
A protected member is visible inside the class and its subclass, and also to other classes in the same package. 
The protected methods can be defined the same.

\lstinputlisting[language=Java, linerange={1-3,7-7,34-34}, style=codeStyle]{../codes/java/src/encapsulation/core/Person.java}

The field can be accessed from a subclass (which is not necessarily in the same package)

\lstinputlisting[language=Java, linerange={1-7,11-13}, style=codeStyle]{../codes/java/src/encapsulation/core/Student.java}

It can be accessed from classes of the same package

\lstinputlisting[language=Java, linerange={1-7,10-12}, style=codeStyle]{../codes/java/src/encapsulation/core/App2.java}

And cannot be accessed from a class outside the package and not inheriting from their containing class.

\lstinputlisting[language=Java, linerange={1-8,13-13,16-18}, style=codeStyle]{../codes/java/src/encapsulation/main/App.java}

It is also important to point out that protected members can be accessed from another object of the same class 

\lstinputlisting[language=Java, linerange={3-3,7-7,29-29,31-34}, style=codeStyle]{../codes/java/src/encapsulation/core/Person.java}

\subsection{Javascript} 

There is no protected members in Javascript. 
You can find some hacks on the web, but they can be expensive in term of memory and processing power.
To emulate such mechanism, you have to know the caller object. 
If you insist on having protected and private members, some libraries can be helpful such as \nameword{mozart}\footnote{mozart: \url{https://github.com/philipwalton/mozart}}.

% https://philipwalton.com/articles/implementing-private-and-protected-members-in-javascript/
% https://philipwalton.com/articles/implementing-private-and-protected-members-in-javascript/

\subsection{Lua}

There is no protected members in Lua, but you can use a naming convention to specify them.

\subsection{Perl}

Perl can capture the caller (the location or package where a function was called) using the keyword \keyword{caller}. 
Also, the base class \keyword{UNIVERSAL} provides a method \keyword{isa} which returns true if the object is blessed to the given type or inherits from it.
Adding to it the keyword \keyword{\_\_PACKAGE\_\_} which captures the current package and exception management \keyword{die}, Bingo! we can limit access. 
We cannot limit visibility: the method is still visible, but we raise an exception if the access does not serve our purpose.

\begin{lstlisting}[language=Perl, style=codeStyle]
package Person;
# ...
sub protected_fct {
	die "protected_fct is protected!" unless caller->isa(__PACKAGE__);
	# ... The rest of the code
}
1;
\end{lstlisting} 

As for fields, you can hide them entirely (private) then use getters and setters to access them. 
Since these getters/setters are methods, you can limit their access. 

%\lstinputlisting[language=Perl, linerange={4-4}, style=codeStyle]{../codes/perl/encapsulation/person.pl}

\subsection{PHP}

Protected members are prefixed, each, with the keyword \keyword{protected}.

\lstinputlisting[language=PHP, linerange={2-2,6-6,31-31}, style=codeStyle]{../codes/php/encapsulation/person.php}

They can be accessed from the class itself or its subclasses

\lstinputlisting[language=PHP, linerange={5-9,13-15}, style=codeStyle]{../codes/php/encapsulation/student.php}

They cannot be accessed elsewhere

\lstinputlisting[language=PHP, linerange={3-3,6-6,15-15}, style=codeStyle]{../codes/php/encapsulation/app.php}

An object can access a protected member of another of the same class

\lstinputlisting[language=PHP, linerange={2-2,26-26,28-31}, style=codeStyle]{../codes/php/encapsulation/person.php}

\subsection{Python}

In Python, all class members are publicly visible. 
But, Python community use coding conversions to tell developers about a field's visibility. 
Some developers use underscore \keyword{\_} before the class member to signal that it is protected. 
However, looking to this convension's definition in Python's docs: "\textit{a name prefixed with an underscore should be treated as a non-public part of the API. It should be considered an implementation detail and subject to change without notice.}", this suggests that it is a private member convention.
But, since there is another convention for private members (double underscore), we can view one underscore as protected.

\lstinputlisting[language=Python, linerange={4-5,8-8,11-11}, style=codeStyle]{../codes/python/encapsulation/person.py}

It can be accessed from subclasses

\lstinputlisting[language=Python, linerange={4-10}, style=codeStyle]{../codes/python/encapsulation/student.py}

Or, from any class, module or piece of code 

\lstinputlisting[language=Python, linerange={4-7,17-17}, style=codeStyle]{../codes/python/encapsulation/app.py}

The idea is to tell developers that this class member is protected and it can be changed in the future, not to enforce the visibility on them. 
If they want to access it directly, its their choice.

\subsection{Ruby}

All fields are protected by default and there is no other visibility mode. 
Methods are public by default, but can be defined as protected using either \keyword{private} or \keyword{protected}. 

\lstinputlisting[language=Ruby, linerange={2-2,7-8,10-13,16-16,25-33,47-47}, style=codeStyle]{../codes/ruby/encapsulation/person.rb}

The methods defined using \keyword{private} or \keyword{protected} can both be accessed from sub-classes.

\lstinputlisting[language=Ruby, linerange={3-4,13-19}, style=codeStyle]{../codes/ruby/encapsulation/student.rb}

The difference is that those with \keyword{private} cannot be accessed from another object even if it is of the same class. 
Those with \keyword{protected} can be accessed from the an object of the same class.

\lstinputlisting[language=Ruby, linerange={2-2,42-47}, style=codeStyle]{../codes/ruby/encapsulation/person.rb}

The fields and the protected methods cannot be accessed outside the class and its sub-classes

\lstinputlisting[language=Ruby, linerange={6-6,10-11,17-18}, style=codeStyle]{../codes/ruby/encapsulation/app.rb}

\section{Private members}

Fields are set to be private so they can not be changed directly, instead setters and getters are more appropriate for that. 
Private methods are a good way to break tasks into smaller pieces of code. 
Also, they can prevent duplication in case many other public functions needs to do the same thing in some point.

\subsection{C++}

Private members are defined inside the class header using the modifier \keyword{private}. 
Every member coming after that modifier is considered as private.

\lstinputlisting[language=C++, linerange={6-7,21-25}, style=codeStyle]{../codes/cpp/encapsulation/person.h}

These members can be accessed only from the class itself and not its subclasses

\lstinputlisting[language=C++, linerange={3-4,7-8}, style=codeStyle]{../codes/cpp/encapsulation/student.cpp}

Also, an object can access another object's members if they are of the same class whatever their visibility is.

\lstinputlisting[language=C++, linerange={23-27}, style=codeStyle]{../codes/cpp/encapsulation/person.cpp}

\subsection{Java}

Private members are prefixed each by the keyword \keyword{private}. 
A private member (field or method) is visible only inside the class. 

\lstinputlisting[language=Java, linerange={1-3,9-9,34-34}, style=codeStyle]{../codes/java/src/encapsulation/core/Person.java}

It cannot be accessed from a subclass

\lstinputlisting[language=Java, linerange={3-3,5-6,10-13}, style=codeStyle]{../codes/java/src/encapsulation/core/Student.java}

Also, it cannot be accessed from another class

\lstinputlisting[language=Java, linerange={5-8,12-12,16-18}, style=codeStyle]{../codes/java/src/encapsulation/main/App.java}

It is also important to point out that private members can be accessed from another object of the same class 

\lstinputlisting[language=Java, linerange={3-3,7-7,29-30,32-34}, style=codeStyle]{../codes/java/src/encapsulation/core/Person.java}

\subsection{Javascript} 

There is a convention to use underscore \keyword{\_} as a mean to signal a class member as private. 
But the member still accessible as public. 
Another way is to use closures (\textit{the combination of a function and the lexical environment within which that function was declared}), by defining a field inside the constructor using the keyword \keyword{var} or \keyword{let}.
But, any method using this variable must be defined inside the constructor.
This means each time you create a new instance (object), a new method will be created for this object. 

\lstinputlisting[linerange={1-3,5-14}, style=codeStyle]{../codes/javascript/encapsulation/person.js}

The private member defined by convention can be accessed anywhere. 
In contrary, the member defined with closures cannot be accessed.

\lstinputlisting[linerange={1-1,4-4,8-9}, style=codeStyle]{../codes/javascript/encapsulation/app.js}

\subsection{Lua}

Using closures, you can create internal fields inside the constructor of an object.
The functions which access this field must be defined inside the constructor.

\lstinputlisting[language={[5.2]Lua}, linerange={1-6,8-12,19-19}, style=codeStyle]{../codes/lua/encapsulation/person.lua}

You cannot access it outside the class. 
But, you can create a new public field dynamically which will not replace the internal one.

\lstinputlisting[language={[5.2]Lua}, linerange={1-3,7-8}, style=codeStyle]{../codes/lua/encapsulation/app.lua}

\subsection{Perl}

There is a convention to start private members with underscore \keyword{\_}, but they still are accessible. 
For methods, you can use \keyword{caller}, \keyword{eq}, \keyword{\_\_PACKAGE\_\_} and \keyword{die} to limit access though they still are visible. 
If the method is not called from inside the package (which is the class in our case), it will raise an exception and stop executing.

\begin{lstlisting}[language=Perl, style=codeStyle]
package Person;
# ...
sub private_fct {
die "private_fct is private!" unless caller eq __PACKAGE__;
# ... The rest of the code
}
1;
\end{lstlisting} 

As for fields, you can hide them entirely (private) using many ways. 

\subsubsection{Using scope and object reference} 

This is a method I thought of in order to hide internal fields of an object. 
Using a container hash, each object's reference is considered as a key and the value is another hash containing private fields names and their values. 
The hash must be defined using \keyword{my} and not \keyword{our} so it will not be accessible outside the package. 
You can put your package inside a code block so the container hash will not be accessible to the rest of code in the same file.
You can consider this hash as a private static member. 
To get an object's reference in memory, you can use \keyword{refaddr} function. 

\lstinputlisting[language=Perl, linerange={2-4,6-6,8-16,22-23,56-57}, style=codeStyle]{../codes/perl/encapsulation/private.pl}

Then, you can access any field of an object using the instruction \textbf{\$private\{refaddr \$obj\}->\{name\}}.
You can, also, create a private getter/setter if you do not want to repeat that instruction every time you want to access a private field. 

\lstinputlisting[language=Perl, linerange={2-5,25-32,43-47,56-57}, style=codeStyle]{../codes/perl/encapsulation/private.pl}

If you want a protected access, just modify the setter/getter to do so. 
In addition, if you want to prevent modifying the content of a protected field from a subclass, just delete the setter code block.

\subsubsection{Using closures} 

This solution is what you will find on the web. %ref to https://www.perlmonks.org/?node_id=8251
Instead of using hashes as objects, you may implement them as closures. 
A closure is a subroutine reference that has access to the lexical variables that were in scope when it was created.
The idea is to define a subroutine that will act as an setter/getter method and bless it into the class instead of the hash. 
The solution presented here is a simplified one, with some restrictions on closure. 
Here I forbid access outside the class to make all fields private, then you can create second hand setters/getters for those you want to grant access to.
 
\lstinputlisting[language=Perl, linerange={1-28,49-49}, style=codeStyle]{../codes/perl/encapsulation/Person.pm}


\subsection{PHP}

Private members are prefixed, each, with the keyword \keyword{private}.

\lstinputlisting[language=PHP, linerange={2-2,7-7,31-31}, style=codeStyle]{../codes/php/encapsulation/person.php}

They can be accessed only from the class itself and not its subclasses

\lstinputlisting[language=PHP, linerange={5-8,12-15}, style=codeStyle]{../codes/php/encapsulation/student.php}

They cannot be accessed elsewhere

\lstinputlisting[language=PHP, linerange={3-3,6-6,14-14}, style=codeStyle]{../codes/php/encapsulation/app.php}

An object can access a protected member of another of the same class

\lstinputlisting[language=PHP, linerange={2-2,26-27,29-31}, style=codeStyle]{../codes/php/encapsulation/person.php}

\subsection{Python}

As said before: all members are public (A remainder so you don't think otherwise).
But, there is a mechanism used to avoid name clashes of names with names defined by subclasses, called \nameword{name mangling}.
When a field is prefixed by a double underscore \keyword{\_\_}, its name will be changed to \textbf{\_classname\_\_membername}.

\lstinputlisting[language=Python, linerange={4-8,10-10}, style=codeStyle]{../codes/python/encapsulation/person.py}

It can be accessed from subclasses using the new mangled name

\lstinputlisting[language=Python, linerange={4-9,11-12}, style=codeStyle]{../codes/python/encapsulation/student.py}

Or, from any class, module or piece of code. 
Trying to read the field as it is defined in the class will raise an error since it does not exist with that name anymore.
While trying to assign a value to the field will work, because you are actually creating a new field of the object dynamically.

\lstinputlisting[language=Python, linerange={4-7,14-16}, style=codeStyle]{../codes/python/encapsulation/app.py}

\subsection{Ruby}

There are no private members in Ruby, which are visible just inside their class. 
Even if you try to limit access of a field to read-only, it can be accessed from sub-classes. 
This is because the field itself is protected and \keyword{attr\_reader} creates a getter for this field. 

\lstinputlisting[language=Ruby, linerange={2-2,7-8,10-11,16-16,47-47}, style=codeStyle]{../codes/ruby/encapsulation/person.rb}

\lstinputlisting[language=Ruby, linerange={3-3,5-6,9-9,19-19}, style=codeStyle]{../codes/ruby/encapsulation/student.rb}

\section{Other visibility modes}

\subsection{C++}

public, protected and private are the only visibility modes applied to class members. 
But, a class can specify an external function of another class to be able to access its members whatever their visibility mode is. 
This is known as \nameword{friend function} and \nameword{friend class} respectively. 
They can be specified using the keyword \keyword{friend} before each. 

\lstinputlisting[language=C++, linerange={6-11,18-19,21-22,25-25}, style=codeStyle]{../codes/cpp/encapsulation/person.h}

The new class can access all fields 

\lstinputlisting[language=C++, linerange={7-14}, style=codeStyle]{../codes/cpp/encapsulation/frnd.h}

The same thing for the function

\lstinputlisting[language=C++, linerange={5-8}, style=codeStyle]{../codes/cpp/encapsulation/app.cpp}

There are some notes about friendship:
\begin{itemize}
	\item It is not transitive: a friend of a friend is not a friend.
	\item It can not be inherited: subclasses of a friend class are not friends.
\end{itemize}

\subsection{Java}

Another visibility mode is \nameword{package visibility} where the members of a class are only accessed from classes of the same package. 
If you don't prefix the member by any of the three visibility modes, it means it is a package member.

\lstinputlisting[language=Java, linerange={1-3,6-6,34-34}, style=codeStyle]{../codes/java/src/encapsulation/core/Person.java}

It can be accessed from classes of the same package

\lstinputlisting[language=Java, linerange={1-6,8-8,10-12}, style=codeStyle]{../codes/java/src/encapsulation/core/App2.java}

But cannot be accessed outside the package

\lstinputlisting[language=Java, linerange={1-8,11-11,16-18}, style=codeStyle]{../codes/java/src/encapsulation/main/App.java}

%\subsection{Javascript} 
%
%\lstinputlisting[linerange={1-1,5-5}, style=codeStyle]{../codes/javascript/person2.js}
%
%\lstinputlisting[linerange={1-1,15-15}, style=codeStyle]{../codes/javascript/person.js}
%
%
%\subsection{Lua}
%
%\lstinputlisting[language={[5.2]Lua}, linerange={1-2}, style=codeStyle]{../codes/lua/person.lua}
%
%\subsection{Perl}
%
%\lstinputlisting[language=Perl, linerange={4-4}, style=codeStyle]{../codes/perl/person.pl}
%
%\subsection{PHP}
%
%\lstinputlisting[language=PHP, linerange={2-2,27-27}, style=codeStyle]{../codes/php/person.php}
%
%\subsection{Python}
%
%\lstinputlisting[language=Python, linerange={4-4}, style=codeStyle]{../codes/python/person.py}
%
%\subsection{Ruby}
%
%\lstinputlisting[language=Ruby, linerange={3-3,19-19}, style=codeStyle]{../codes/ruby/person.rb}


%=====================================================================
\ifx\wholebook\relax\else
% \cleardoublepage
% \bibliographystyle{../use/ESIbib}
% \bibliography{../bib/RATstat}
	\end{document}
\fi
%=====================================================================