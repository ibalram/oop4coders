%=====================================================================
\ifx\wholebook\relax\else
	\documentclass{KodeBook}
	\usepackage{amsmath,amssymb}             % AMS Math
\usepackage[utf8]{inputenc}
%\usepackage[T1]{fontenc}

%\usepackage[pdftex]{graphicx}

%\usepackage{listingsutf8}
%\usepackage{xcolor}
%\usepackage{times}
\usepackage{array}
\usepackage{natbib}
\usepackage{lscape}%to flip tables in a page
\usepackage{pdflscape}
\usepackage{longtable}
\usepackage{tabu}
\usepackage{wrapfig}

%\usepackage[english]{babel}

\bibliographystyle{engdnat}%unsrtnat, plainnat

%\usepackage{pgf-umlcd}




\hypersetup{
	colorlinks,
	urlcolor=blue
}

\renewcommand{\UrlFont}{\ttfamily\footnotesize}

\DeclareAcronym{oop}{
	short = OOP ,
	long  = Object-oriented programming ,
	class = abbrev
}

\DeclareAcronym{oo}{
	short = OO ,
	long  = Object-oriented ,
	class = abbrev
}

\DeclareAcronym{ecma}{
	short = ECMA ,
	long  = European Computer Manufacturers Association ,
	class = abbrev
}

\DeclareAcronym{cli}{
	short = CLI ,
	long  = Command-Line Interface ,
	class = abbrev
}

\DeclareAcronym{cgi}{
	short = CGI ,
	long  = Common Gateway Interface ,
	class = abbrev
}

%\makeglossaries

%\newacronym{oop}{OOP}{Object-oriented programming}
	\begin{document}
\fi
%=====================================================================

\chapter{Encapsulation}

\begin{introduction}
	Encapsulation is one important concept of OOP. 
	It is the mechanism of restricting direct access to some members of a class.
	It helps managing complexity when debugging source code. 
	If a field is accessed everywhere in your source code, this makes it difficult to find errors related to this field. 
	Also, if a field of a class (A) is used by a class (B) then we changed it (either its name or how it gets assigned), we have to change all the code where it appears in (B).
	In this chapter, we will show different access privileges and how they are implemented in each language.
\end{introduction}

\section{Public members}


\section{Friend members}


\section{Protected members}


\section{Private members}




%=====================================================================
\ifx\wholebook\relax\else
% \cleardoublepage
% \bibliographystyle{../use/ESIbib}
% \bibliography{../bib/RATstat}
	\end{document}
\fi
%=====================================================================