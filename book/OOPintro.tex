%=====================================================================
\ifx\wholebook\relax\else
	\documentclass[12pt]{book}
	\usepackage{amsmath,amssymb}             % AMS Math
\usepackage[utf8]{inputenc}
%\usepackage[T1]{fontenc}

%\usepackage[pdftex]{graphicx}

%\usepackage{listingsutf8}
%\usepackage{xcolor}
%\usepackage{times}
\usepackage{array}
\usepackage{natbib}
\usepackage{lscape}%to flip tables in a page
\usepackage{pdflscape}
\usepackage{longtable}
\usepackage{tabu}
\usepackage{wrapfig}

%\usepackage[english]{babel}

\bibliographystyle{engdnat}%unsrtnat, plainnat

%\usepackage{pgf-umlcd}




\hypersetup{
	colorlinks,
	urlcolor=blue
}

\renewcommand{\UrlFont}{\ttfamily\footnotesize}

\DeclareAcronym{oop}{
	short = OOP ,
	long  = Object-oriented programming ,
	class = abbrev
}

\DeclareAcronym{oo}{
	short = OO ,
	long  = Object-oriented ,
	class = abbrev
}

\DeclareAcronym{ecma}{
	short = ECMA ,
	long  = European Computer Manufacturers Association ,
	class = abbrev
}

\DeclareAcronym{cli}{
	short = CLI ,
	long  = Command-Line Interface ,
	class = abbrev
}

\DeclareAcronym{cgi}{
	short = CGI ,
	long  = Common Gateway Interface ,
	class = abbrev
}

%\makeglossaries

%\newacronym{oop}{OOP}{Object-oriented programming}
	\begin{document}
		\chapter{Introduction to OOP}
\fi
%=====================================================================

\begin{introduction}
	In this chapter, we will present a little history followed by the concepts of OOP.
	Then, we present the benefits of OOP which are arguable whether OOP nowadays has achieve them or not. 
	You can argue that OOP has not afford some benefits it meant to do. 
	The purpose of this book is to be a guide to programmers wanting to compare different implementations of this paradigm. 
	To see how OOP is perceived, we present some point of views from both sides: with and against.
\end{introduction}

\section{History}

\section{Concepts}

\subsection{Objects and classes}

\subsection{Encapsulation}

\subsection{Associations}

\subsection{Inheritance}

\subsection{Polymorphism}


\section{Styles}

\subsection{Class-based}

\subsection{Prototype-based}

%Lua does not have the concept of class; each object defines its own behavior and has a shape of its own. Nevertheless, it is not difficult to emulate classes in Lua, following the lead from prototype-based languages, such as Self and NewtonScript. In those languages, objects have no classes. Instead, each object may have a prototype, which is a regular object where the first object looks up any operation that it does not know about. To represent a class in such languages, we simply create an object to be used exclusively as a prototype for other objects (its instances). Both classes and prototypes work as a place to put behavior to be shared by several objects.

\section{benefits}


	
\section{Some opinions about OOP}

\subsection{With}

\subsection{Against}

Joe Armstrong, the principal inventor of Erlang:
\begin{quote}
The problem with object-oriented languages is they've got all this implicit environment that they carry around with them. 
You wanted a banana but what you got was a gorilla holding the banana and the entire jungle.
\end{quote}


%=====================================================================
\ifx\wholebook\relax\else
% \cleardoublepage
% \bibliographystyle{../use/ESIbib}
% \bibliography{../bib/RATstat}
	\end{document}
\fi
%=====================================================================