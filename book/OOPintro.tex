%=====================================================================
\ifx\wholebook\relax\else
	\documentclass[12pt]{book}
	\usepackage{amsmath,amssymb}             % AMS Math
\usepackage[utf8]{inputenc}
%\usepackage[T1]{fontenc}
\usepackage{hyperref}
%\usepackage[pdftex]{graphicx}
\usepackage{listings}
%\usepackage{listingsutf8}
%\usepackage{xcolor}
%\usepackage{times}
\usepackage{array}
%\usepackage{natbib}
%\usepackage[left=2.8cm,right=2.2cm,top=2.8cm,bottom=2.8cm,includefoot,includehead,headheight=13.6pt]{geometry}

%\bibliographystyle{ACM-Reference-Format-Journals}


\lstdefinestyle{codeStyle}{
  belowcaptionskip=1\baselineskip,
  breaklines=true,
  frame=L,
  xleftmargin=1.5cm,%\parindent,
  showstringspaces=false,
  basicstyle=\scriptsize\ttfamily\bfseries,
  keywordstyle=\bfseries\color{green!40!black},
  commentstyle=\itshape\color{purple!40!black},
  identifierstyle=\color{blue},
  stringstyle=\color{orange},
  backgroundcolor=\color{blue!10!white},
  lineskip=.2em,
  numbers=left
%  frameround=tttt
}

\lstdefinestyle{shellStyle}{
	belowcaptionskip=1\baselineskip,
	breaklines=true,
	frame=shadwbox,
	rulesepcolor=\color{blue},
	xleftmargin=1.25cm,
	basicstyle=\scriptsize\ttfamily\bfseries\color{green},
	backgroundcolor=\color{black},
	lineskip=.2em,
%	morekeywords={sudo},keywordstyle=\color{red},
	%  frameround=tttt
}



\hypersetup{
	colorlinks,
	urlcolor=blue
}

\renewcommand{\UrlFont}{\ttfamily\footnotesize}

	\begin{document}
		\chapter{Introduction to OOP}
\fi
%=====================================================================

\begin{introduction}
	In this chapter, we will present a little history followed by the concepts of OOP.
	Then, we present the benefits of OOP which are arguable whether OOP nowadays has achieve them or not. 
	You can argue that OOP has not afford some benefits it meant to do. 
	The purpose of this book is to be a guide to programmers wanting to compare different implementations of this paradigm. 
	To see how OOP is perceived, we present some point of views from both sides: with and against.
\end{introduction}


\section{Concepts}

\subsection{Objects and classes}

\subsection{Encapsulation}

\subsection{Associations}

\subsection{Inheritance}

\subsection{Polymorphism}


\section{Styles}

\subsection{Class-based}

\subsection{Prototype-based}
	
\section{Some opinions about OOP}

\subsection{With}

\subsection{Against}

Joe Armstrong, the principal inventor of Erlang:
\begin{quote}
The problem with object-oriented languages is they've got all this implicit environment that they carry around with them. 
You wanted a banana but what you got was a gorilla holding the banana and the entire jungle.
\end{quote}


%=====================================================================
\ifx\wholebook\relax\else
% \cleardoublepage
% \bibliographystyle{../use/ESIbib}
% \bibliography{../bib/RATstat}
	\end{document}
\fi
%=====================================================================