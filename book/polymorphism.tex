%=====================================================================
\ifx\wholebook\relax\else
	\documentclass{KodeBook}
	\usepackage{amsmath,amssymb}             % AMS Math
\usepackage[utf8]{inputenc}
%\usepackage[T1]{fontenc}
\usepackage{hyperref}
%\usepackage[pdftex]{graphicx}
\usepackage{listings}
%\usepackage{listingsutf8}
%\usepackage{xcolor}
%\usepackage{times}
\usepackage{array}
%\usepackage{natbib}
%\usepackage[left=2.8cm,right=2.2cm,top=2.8cm,bottom=2.8cm,includefoot,includehead,headheight=13.6pt]{geometry}

%\bibliographystyle{ACM-Reference-Format-Journals}


\lstdefinestyle{codeStyle}{
  belowcaptionskip=1\baselineskip,
  breaklines=true,
  frame=L,
  xleftmargin=1.5cm,%\parindent,
  showstringspaces=false,
  basicstyle=\scriptsize\ttfamily\bfseries,
  keywordstyle=\bfseries\color{green!40!black},
  commentstyle=\itshape\color{purple!40!black},
  identifierstyle=\color{blue},
  stringstyle=\color{orange},
  backgroundcolor=\color{blue!10!white},
  lineskip=.2em,
  numbers=left
%  frameround=tttt
}

\lstdefinestyle{shellStyle}{
	belowcaptionskip=1\baselineskip,
	breaklines=true,
	frame=shadwbox,
	rulesepcolor=\color{blue},
	xleftmargin=1.25cm,
	basicstyle=\scriptsize\ttfamily\bfseries\color{green},
	backgroundcolor=\color{black},
	lineskip=.2em,
%	morekeywords={sudo},keywordstyle=\color{red},
	%  frameround=tttt
}



\hypersetup{
	colorlinks,
	urlcolor=blue
}

\renewcommand{\UrlFont}{\ttfamily\footnotesize}

	\begin{document}
\fi
%=====================================================================

\chapter{Polymorphism}

\begin{introduction}
	\lettrine{P}{olymorphism} is another key concept of \ac{oop}. 
	In general, the term refers to the ability to assign a different meaning or usage to something in many contexts.
	Hence, it is not a concept specific to \ac{oop} since you can find other forms such as parametric polymorphism (generic programming). 
	In languages with type inheritance, an object of a class can be considered as an object of its superclass; and therefore, the same applications can be executed on it.
	Some languages afford functionalities to verify if a certain object is instantiated from a class, to get its class or to cast the variable using it to another subclass.
	Methods can express polymorphism in two ways: overloading (same name with multiple signatures) and overriding (same signature, different implementation). 
	Not every language affords overloading, but methods overriding is a core task of \ac{oop}.
	Beside \ac{oop} polymorphism, some languages afford some other forms of polymorphism.
\end{introduction} 

\section{Subtype polymorphism}

According to the \nameword{Liskov substitution principle} \citep{1987-liskov}, a function written to take an object of a certain type \textbf{T} must work correctly if passed an object of a type \textbf{S} which is a subtype of \textbf{T}.
C++ and Java are examples of \ac{oop} languages with static type checking affording this principle. 
Some languages do not need an object to be created from a class/prototype extending another to execute certain treatments on it. 
They use a mechanism called \nameword{Duck typing} which states that ``\textit{If it walks like a duck and talks like a duck, it must be a duck}". 

A simple example will be sufficient to show how every language handles such polymorphism. 
Having a class \textbf{Person} with a method \textbf{talk} and two subclasses \textbf{Student} and \textbf{Professor}:
\begin{itemize}
	\item A function \textbf{announce} designed to call \textbf{talk} of \textbf{Person} is passed objects of \textbf{Student} and \textbf{Professor} as argument.
	\item A table intended to contain objects of \textbf{Person} is filled with objects of its subtypes.
	\item If the language supports Duck typing: some unrelated classes and objects having the method \textbf{talk} can be used with the two previous propositions.
\end{itemize}

\subsection{C++}

Lets create our classes

\lstinputlisting[language={[KB]C++}, linerange={4-13}, style=codeStyle]{../codes/cpp/polymorphism/subtype.cpp}

Passing objects of these three classes to a function designed for objects of type \textbf{Person} will work just fine. 

\lstinputlisting[language={[KB]C++}, linerange={15-26, 31-33}, style=codeStyle]{../codes/cpp/polymorphism/subtype.cpp}

Objects of type \textbf{Person} or any other subclass can be assigned to a variable of type \textbf{Person}.

\lstinputlisting[language={[KB]C++}, linerange={20-23, 28-33}, style=codeStyle]{../codes/cpp/polymorphism/subtype.cpp}

\subsection{Java}

Lets create our classes

\lstinputlisting[language={[KB]Java}, linerange={35-40,45-47}, style=codeStyle]{../codes/java/src/polymorphism/Subtype.java}

Passing objects of these three classes to a function designed for objects of type \textbf{Person} will work just fine. 

\lstinputlisting[language={[KB]Java}, linerange={3-7, 13-20, 32-33}, style=codeStyle]{../codes/java/src/polymorphism/Subtype.java}

If we use an interface (\textbf{Machine}) and a function \textbf{announce2} similar to \textbf{announce} but taking as parameter \textbf{Machine} instead of \textbf{Person}, this will work as well because interfaces in Java define new types and support type polymorphism.

\lstinputlisting[language={[KB]Java}, linerange={3-3, 8-8, 13-13, 26-34, 42-45, 50-56}, style=codeStyle]{../codes/java/src/polymorphism/Subtype.java}

Objects of type \textbf{Person} or any other subclass can be assigned to a variable of type \textbf{Person}.

\lstinputlisting[language={[KB]Java}, linerange={3-3, 13-16,22-24,32-33}, style=codeStyle]{../codes/java/src/polymorphism/Subtype.java}


%
%\subsection{Javascript}
%
%\lstinputlisting[linerange={3-4}, style=codeStyle]{../javascript/person.js}
%
%\lstinputlisting[linerange={17-17}, style=codeStyle]{../javascript/person.js}
%
%Methods: 
%
%\lstinputlisting[linerange={1-1,8-15}, style=codeStyle]{../javascript/person.js}
%
%\lstinputlisting[linerange={9-15}, style=codeStyle]{../javascript/person2.js}
%
%getters and setters can be defined using keywords \textbf{get} and \textbf{set} in ES6.
%
%\subsection{Lua}
%
%\lstinputlisting[language={[5.2]Lua}, linerange={4-9}, style=codeStyle]{../lua/person.lua}
%
%\lstinputlisting[language={[5.2]Lua}, linerange={11-17}, style=codeStyle]{../lua/person.lua}
%
%\lstinputlisting[language={[5.2]Lua}, linerange={22-24}, style=codeStyle]{../lua/person.lua}
%
%
%\subsection{Perl}
%
%\lstinputlisting[language=Perl, linerange={10-13}, style=codeStyle]{../perl/person.pl}
%
%\lstinputlisting[language=Perl, linerange={6-6}, style=codeStyle]{../perl/person.pl}
%
%Methods
%
%\lstinputlisting[language=Perl, linerange={25-32}, style=codeStyle]{../perl/person.pl}
%
%\lstinputlisting[language=Perl, linerange={39-41}, style=codeStyle]{../perl/person.pl}

%\subsection{PHP}
%
%\lstinputlisting[language=PHP, linerange={2-7,27-27}, style=codeStyle]{../php/person.php}
%
%\lstinputlisting[language=PHP, linerange={2-2,19-25,27-27}, style=codeStyle]{../php/person.php}
%
%\lstinputlisting[language=PHP, linerange={32-34}, style=codeStyle]{../php/person.php}
%
%\subsection{Python}
%
%\lstinputlisting[language=Python, linerange={4-11}, style=codeStyle]{../python/person.py}
%
%\subsection{Ruby}
%
%\lstinputlisting[language=Ruby, linerange={3-10,19-19}, style=codeStyle]{../ruby/person.rb}
%
%\lstinputlisting[language=Ruby, linerange={3-3,12-19}, style=codeStyle]{../ruby/person.rb}
%
%\lstinputlisting[language=Ruby, linerange={24-26}, style=codeStyle]{../ruby/person.rb}

\section{Type manipulation}

\begin{itemize}
	\item Get object type
	\item Is instance of
	\item casting
\end{itemize}


\section{Methods Overloading}


\section{Methods Overriding}



%\subsection{C++}
%
%\lstinputlisting[language=C++, style=codeStyle]{../cpp/type.cpp}
%
%
%\subsection{Java}
%
%\lstinputlisting[language=Java, style=codeStyle]{../java/src/Type.java}
%
%
%\subsection{Javascript}
%
%\lstinputlisting[linerange={3-4}, style=codeStyle]{../javascript/person.js}
%
%\lstinputlisting[linerange={17-17}, style=codeStyle]{../javascript/person.js}
%
%Methods: 
%
%\lstinputlisting[linerange={1-1,8-15}, style=codeStyle]{../javascript/person.js}
%
%\lstinputlisting[linerange={9-15}, style=codeStyle]{../javascript/person2.js}
%
%getters and setters can be defined using keywords \textbf{get} and \textbf{set} in ES6.
%
%\subsection{Lua}
%
%\lstinputlisting[language={[5.2]Lua}, linerange={4-9}, style=codeStyle]{../lua/person.lua}
%
%\lstinputlisting[language={[5.2]Lua}, linerange={11-17}, style=codeStyle]{../lua/person.lua}
%
%\lstinputlisting[language={[5.2]Lua}, linerange={22-24}, style=codeStyle]{../lua/person.lua}
%
%
%\subsection{Perl}
%
%\lstinputlisting[language=Perl, linerange={10-13}, style=codeStyle]{../perl/person.pl}
%
%\lstinputlisting[language=Perl, linerange={6-6}, style=codeStyle]{../perl/person.pl}
%
%Methods
%
%\lstinputlisting[language=Perl, linerange={25-32}, style=codeStyle]{../perl/person.pl}
%
%\lstinputlisting[language=Perl, linerange={39-41}, style=codeStyle]{../perl/person.pl}

%\subsection{PHP}
%
%\lstinputlisting[language=PHP, linerange={2-7,27-27}, style=codeStyle]{../php/person.php}
%
%\lstinputlisting[language=PHP, linerange={2-2,19-25,27-27}, style=codeStyle]{../php/person.php}
%
%\lstinputlisting[language=PHP, linerange={32-34}, style=codeStyle]{../php/person.php}
%
%\subsection{Python}
%
%\lstinputlisting[language=Python, linerange={4-11}, style=codeStyle]{../python/person.py}
%
%\subsection{Ruby}
%
%\lstinputlisting[language=Ruby, linerange={3-10,19-19}, style=codeStyle]{../ruby/person.rb}
%
%\lstinputlisting[language=Ruby, linerange={3-3,12-19}, style=codeStyle]{../ruby/person.rb}
%
%\lstinputlisting[language=Ruby, linerange={24-26}, style=codeStyle]{../ruby/person.rb}

\section{Not OOP related polymorphism}

\subsection{Operator overloading}

\subsection{Generic programming}

%=====================================================================
\ifx\wholebook\relax\else
% \cleardoublepage
% \bibliographystyle{../use/ESIbib}
% \bibliography{../bib/RATstat}
	\end{document}
\fi
%=====================================================================