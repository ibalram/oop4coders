%=====================================================================
\ifx\wholebook\relax\else
	\documentclass{KBook}
	\usepackage{amsmath,amssymb}             % AMS Math
\usepackage[utf8]{inputenc}
%\usepackage[T1]{fontenc}
\usepackage{hyperref}
%\usepackage[pdftex]{graphicx}
\usepackage{listings}
%\usepackage{listingsutf8}
%\usepackage{xcolor}
%\usepackage{times}
\usepackage{array}
%\usepackage{natbib}
%\usepackage[left=2.8cm,right=2.2cm,top=2.8cm,bottom=2.8cm,includefoot,includehead,headheight=13.6pt]{geometry}

%\bibliographystyle{ACM-Reference-Format-Journals}


\lstdefinestyle{codeStyle}{
  belowcaptionskip=1\baselineskip,
  breaklines=true,
  frame=L,
  xleftmargin=1.5cm,%\parindent,
  showstringspaces=false,
  basicstyle=\scriptsize\ttfamily\bfseries,
  keywordstyle=\bfseries\color{green!40!black},
  commentstyle=\itshape\color{purple!40!black},
  identifierstyle=\color{blue},
  stringstyle=\color{orange},
  backgroundcolor=\color{blue!10!white},
  lineskip=.2em,
  numbers=left
%  frameround=tttt
}

\lstdefinestyle{shellStyle}{
	belowcaptionskip=1\baselineskip,
	breaklines=true,
	frame=shadwbox,
	rulesepcolor=\color{blue},
	xleftmargin=1.25cm,
	basicstyle=\scriptsize\ttfamily\bfseries\color{green},
	backgroundcolor=\color{black},
	lineskip=.2em,
%	morekeywords={sudo},keywordstyle=\color{red},
	%  frameround=tttt
}



\hypersetup{
	colorlinks,
	urlcolor=blue
}

\renewcommand{\UrlFont}{\ttfamily\footnotesize}

	\begin{document}
	\chapter{Hello World}
\fi
%=====================================================================

%Introduction


\section{Getting started}

\subsection{C++}

To use C++ you have to install its compiler. 
The most used one is GNU g++\footnote{GNU g++: \url{gcc.gnu.org/}} which is distributed freely and available for most systems.

\subsubsection{Linux}
Simply install g++ on your system; on Ubuntu you, simply, type:
\begin{verbatim}
>> sudo install g++
\end{verbatim}

\subsubsection{Windows}

If you want to use g++ on Windows, you can install:
\begin{itemize}
	\item MinGW: \url{http://www.mingw.org} 
	\item CygWin: \url{http://www.cygwin.com} if you want to use Linux on Windows
\end{itemize}

You can use Microsoft Visual studio: \url{https://www.visualstudio.com/vs/visual-studio-express/}

Also, you can install Turbo C++ on \url{https://turboc.codeplex.com}

\subsection{Java}

Install OracleJDK or OpenJDK

\subsection{Javascript}

Install node

\subsection{Lua}

Install lua

\subsection{Perl}

Install perl

\subsection{PHP}

install php

\subsection{Python}

install python

\subsection{Ruby}

install ruby

\section{Functions}
Let take the factorial function as an example. 

\subsection{C++}

\lstinputlisting[language=C++, style=codeStyle]{../cpp/func.cpp}
%\lstinputlisting[language=Python, firstline=37, lastline=45]{source_filename.py}

\subsection{Java}

\lstinputlisting[language=Java, style=codeStyle]{../java/src/Func.java}

\subsection{Javascript}

\lstinputlisting[style=codeStyle]{../javascript/func.js}

\subsection{Lua}

\lstinputlisting[language={[5.2]Lua}, style=codeStyle]{../lua/func.lua}

\subsection{Perl}

\lstinputlisting[language=Perl, style=codeStyle]{../perl/func.pl}

\subsection{PHP}

\lstinputlisting[language=PHP, style=codeStyle]{../php/func.php}

\subsection{Python}

\lstinputlisting[language=Python, style=codeStyle]{../python/func.py}

\subsection{Ruby}


\section{Hello world}

\subsection{C++}

\lstset{language=C++}
\lstinputlisting[language=C++, style=codeStyle]{../cpp/helloworld.cpp}
%\lstinputlisting[language=Python, firstline=37, lastline=45]{source_filename.py}

\subsection{Java}

\lstinputlisting[language=Java, style=codeStyle]{../java/src/HelloWorld.java}

\subsection{Javascript}

\lstinputlisting[style=codeStyle, firstline=6, lastline=9]{../javascript/helloworld.htm}

\subsection{Lua}

\lstinputlisting[language={[5.2]Lua}, style=codeStyle]{../lua/helloworld.lua}

\subsection{Perl}

\lstinputlisting[language=Perl, style=codeStyle]{../perl/helloworld.pl}

\subsection{PHP}

\lstinputlisting[language=PHP, style=codeStyle]{../php/helloworld.php}

\subsection{Python}

\lstinputlisting[language=Python, style=codeStyle]{../python/helloworld.py}

\subsection{Ruby}

\lstinputlisting[language=Ruby, style=codeStyle]{../ruby/helloworld.rb}

\section{Functions}
Let take the factorial function as an example. 

\subsection{C++}

\lstinputlisting[language=C++, style=codeStyle]{../cpp/func.cpp}
%\lstinputlisting[language=Python, firstline=37, lastline=45]{source_filename.py}

\subsection{Java}

\lstinputlisting[language=Java, style=codeStyle]{../java/src/Func.java}

\subsection{Javascript}

\lstinputlisting[style=codeStyle]{../javascript/func.js}

\subsection{Lua}

\lstinputlisting[language={[5.2]Lua}, style=codeStyle]{../lua/func.lua}

\subsection{Perl}

\lstinputlisting[language=Perl, style=codeStyle]{../perl/func.pl}

\subsection{PHP}

\lstinputlisting[language=PHP, style=codeStyle]{../php/func.php}

\subsection{Python}

\lstinputlisting[language=Python, style=codeStyle]{../python/func.py}

\subsection{Ruby}

\lstinputlisting[language=Ruby, style=codeStyle]{../ruby/func.rb}

%=====================================================================
\ifx\wholebook\relax\else
% \cleardoublepage
% \bibliographystyle{../use/ESIbib}
% \bibliography{../bib/RATstat}
	\end{document}
\fi
%=====================================================================