%=====================================================================
\ifx\wholebook\relax\else
	\documentclass{KodeBook}
	\usepackage{amsmath,amssymb}             % AMS Math
\usepackage[utf8]{inputenc}
%\usepackage[T1]{fontenc}
\usepackage{hyperref}
%\usepackage[pdftex]{graphicx}
\usepackage{listings}
%\usepackage{listingsutf8}
%\usepackage{xcolor}
%\usepackage{times}
\usepackage{array}
%\usepackage{natbib}
%\usepackage[left=2.8cm,right=2.2cm,top=2.8cm,bottom=2.8cm,includefoot,includehead,headheight=13.6pt]{geometry}

%\bibliographystyle{ACM-Reference-Format-Journals}


\lstdefinestyle{codeStyle}{
  belowcaptionskip=1\baselineskip,
  breaklines=true,
  frame=L,
  xleftmargin=1.5cm,%\parindent,
  showstringspaces=false,
  basicstyle=\scriptsize\ttfamily\bfseries,
  keywordstyle=\bfseries\color{green!40!black},
  commentstyle=\itshape\color{purple!40!black},
  identifierstyle=\color{blue},
  stringstyle=\color{orange},
  backgroundcolor=\color{blue!10!white},
  lineskip=.2em,
  numbers=left
%  frameround=tttt
}

\lstdefinestyle{shellStyle}{
	belowcaptionskip=1\baselineskip,
	breaklines=true,
	frame=shadwbox,
	rulesepcolor=\color{blue},
	xleftmargin=1.25cm,
	basicstyle=\scriptsize\ttfamily\bfseries\color{green},
	backgroundcolor=\color{black},
	lineskip=.2em,
%	morekeywords={sudo},keywordstyle=\color{red},
	%  frameround=tttt
}



\hypersetup{
	colorlinks,
	urlcolor=blue
}

\renewcommand{\UrlFont}{\ttfamily\footnotesize}

	\begin{document}
\fi
%=====================================================================

%Introduction
\chapter{Hello World}

\begin{introduction}
	This chapter is meant to introduce different oriented-object programming languages used in this book. 
	First of all, installing instructions must be presented to help anyone with zero knowledge about these languages. 
	Then, we will present different implementations of these languages from ``Hello world" to functions and exceptions. 
	We have to point out that this chapter introduces the basic syntax without diving in different OOP concepts.
	But, in case of languages with strong OOP, we have to present some concepts such as ``override" in ``exceptions".
\end{introduction}


\section{Getting started}

\subsection{C++}

C++ is a general-purpose compiled programming language which affords object-oriented paradigm among others. 
It was developed by \nameword{Bjarne Stroustrup} at Bell Labs since 1979.

To use C++ you have to install its compiler. 
The most used one is GNU g++\footnote{GNU g++: \url{gcc.gnu.org/}} which is distributed freely and available for most systems.

\subsubsection{Linux}
Simply install g++ on your system; on Ubuntu you, simply, type:
\begin{lstlisting}[style=shellStyle]
$ sudo apt-get update
$ sudo apt-get install g++
\end{lstlisting}

\subsubsection{Windows}

If you want to use g++ on Windows, you can install:
\begin{itemize}
	\item MinGW: \url{http://www.mingw.org} 
	\item CygWin: \url{http://www.cygwin.com} if you want to use Linux on Windows
\end{itemize}

You can use Microsoft Visual studio: \url{https://www.visualstudio.com/vs/visual-studio-express/}

Also, you can install Turbo C++ on \url{https://turboc.codeplex.com}

\subsection{Java}

Java is a general-purpose compiled programming language which is concurrent, class-based, object-oriented. 
It is intended to generate platform-independent programs. 
That is, the same compiled program called bytecodes will be able to execute in different machines and operating systems using a virtual machine. 
Developed by \nameword{James Gosling} at Sun Microsystems (acquired by Oracle Corporation) in 1995.

This is a detailed description on how to install Java \url{https://www.java.com/en/download/help/download_options.xml}.
Simply, to install Oracle JDK, download it from this link: \url{http://www.oracle.com/technetwork/java/javase/downloads/index.html}. 

As for OpenJDK, you can install it on Linux distributions following these instructions: \url{http://openjdk.java.net/install/}. 
On some distributions such as \textbf{Linux Mint}, it is installed by default.
For example, in Ubuntu you type:
\begin{lstlisting}[style=shellStyle]
$ sudo apt-get update
$ sudo apt-get install openjdk-8-jdk
\end{lstlisting}

It is recommended to use an IDE:
\begin{itemize}
	\item Eclipse: which is used in our case. \url{http://www.eclipse.org}
	\item IntelliJ IDEA: \url{https://www.jetbrains.com/idea/}
	\item NetBeans: \url{https://netbeans.org}
\end{itemize}

\subsection{Javascript}

JavaScript is an interpreted programming language for the web, which supports object-oriented paradigm. 
It is meant to be used as client-side programming language executed on different browser, but it can also execute on server side using NodeJs for example.
It was developed by Netscape Communications Corporation, and designed by \nameword{Brendan Eich} in 1995. 

Javascript can be used directly on any browser. 
Suppose we have a file called "func.js" containing a function "fact(n)" which calculates the factorial of a number. 
To use it, all you have to do is linking the file and calling the function in an HTML file:
\begin{lstlisting}[language={[KB]Javascript}, style=codeStyle]
<html>
  <head>
    <title>Functions</title>
      <script type="text/javascript" src="func.js">
      </script>
      <script type="text/javascript">
          function exec(){
              var n = document.getElementById("n").value;
              var res = "Fact(" + n + ")= ";
              res += fact(n);
              document.getElementById("result").innerHTML = res;
          }
      </script>
  </head>
    <body>
        Please enter an integer value: 
        <input type="number" id="n" />
        <button id="func" onclick="exec()">Factorial</button>
        <div id="result"></div>
    </body>
</html>
\end{lstlisting}


Or you can use \nameword{NodeJs} to execute it directly on the shell which is the case in our late examples. 
You can download it here: \url{https://nodejs.org/en/download/}. 
In case of some Linux distributions, you can install it from their repositories:
\begin{lstlisting}[style=shellStyle]
$ sudo apt-get update
$ sudo apt-get install nodejs
\end{lstlisting}

\subsection{Lua}

Lua is an interpreted programming language, which supports object-oriented paradigm. 
It was designed primarily for embedded systems.
Designed by \nameword{Roberto Ierusalimschy}, \nameword{Luiz Henrique de Figueiredo} and \nameword{Waldemar Celes} in 1993.
It is mostly used as a scripting language for games development.

Download it from here: \url{http://lua-users.org/wiki/LuaBinaries}. 
In Ubuntu, you can install it by using its version number as:
\begin{lstlisting}[style=shellStyle]
$ sudo apt-get update
$ sudo apt-get install lua5.2
\end{lstlisting}


\subsection{Perl}

Lua is a general-purpose interpreted programming language, which supports oriented-object paradigm. 
It was developed by \nameword{Larry Wall} in 1987 to simplify report processing on Unix.
It is used as a CGI scripting language for web servers.

To install Perl, please refer to this page: \url{https://www.perl.org/get.html}. 
It is high probable that you have Perl installed on your Linux. 
As for Windows users, there are two projects: 
\begin{itemize}
	\item Strawberry Perl: \url{http://strawberryperl.com}
	\item ActiveState Perl: \url{http://www.activestate.com/activeperl/downloads}
\end{itemize}

\subsection{PHP}

PHP is a general-purpose server-side scripting language which supports oriented-object programming since version 3.0  and improved in version 5.0.
It was originally created by \nameword{Rasmus Lerdorf} in 1994. 

For a detailed description on how to install the server, please refer to this page: \url{http://php.net/manual/en/install.php}.
On Ubuntu, we just need the language:
\begin{lstlisting}[style=shellStyle]
$ sudo apt-get update
$ sudo apt-get install php
\end{lstlisting}
It will install a command-line interface (CLI) called "php"; used to execute a php file directly from the shell.

\subsection{Python}

Python is an interpreted general-purpose programming language, support multiple paradigms including OOP. 
It was created by \nameword{Guido van Rossum} and released in 1991.
It supports code readability as a design philosophy by using indentations as a mean of defining code blocks.

Refer to this page: \url{https://www.python.org/downloads/}. 
For Ubuntu, you can type:
\begin{lstlisting}[style=shellStyle]
$ sudo apt-get update
$ sudo apt-get install python
\end{lstlisting}
Or you can refer to this page for further information: \url{http://docs.python-guide.org/en/latest/starting/install3/linux/}.
You have to verify if it is already installed (Linux Mint has it by default).

\subsection{Ruby}

Ruby is an interpreted, object-oriented, general-purpose programming language. 
It was designed and developed by \nameword{Yukihiro Matsumoto} in 1990.

To install it, please refer to this page: \url{http://www.ruby-lang.org/en/documentation/installation/}. 
On Windows, you can use RubyInstaller: \url{https://rubyinstaller.org}.
On Linux systems, you can find it on their repositories, such as Ubuntu:
\begin{lstlisting}[style=shellStyle]
$ sudo apt-get update
$ sudo apt-get install ruby
\end{lstlisting}
You have to verify if it is already installed (Linux Mint has it by default).

\section{Hello world}

Let's start with a Hello World introductory set of codes. 


\subsection{C++}

A C++ program always has a main function, which represents the interface with the operating system. 
When you call a program (from shell for example), you actually calling this function. 
The main function is defined directly in the file and not inside a class or a structure.
%
\lstinputlisting[language={[KB]C++}, style=codeStyle]{../codes/cpp/hello/helloworld.cpp}
%

Suppose we use g++ for C++. 
To compile this code, you can specify the output file's name or not
\begin{lstlisting}[style=shellStyle]
$ g++ helloworld.cpp 
$ g++ helloworld.cpp -o exec 
\end{lstlisting}
This will create a file \textbf{a.out}, which we can execute as follows
\begin{lstlisting}[style=shellStyle]
$ ./a.out
$ ./exec 
\end{lstlisting}

\subsection{Java}

In Java, a .java file can contain many classes, but only one with a public keyword. 
The public class must have the same name as the file. 
The main function is defined inside the main class.

\lstinputlisting[language={[KB]Java}, style=codeStyle]{../codes/java/src/hello/HelloWorld.java}

To compile this code using command-line
\begin{lstlisting}[style=shellStyle]
$ javac HelloWorld.java
\end{lstlisting}
This will create a file \textbf{HelloWorld.class}, which we can be executed as follows
\begin{lstlisting}[style=shellStyle]
$ java HelloWorld
\end{lstlisting}

\subsection{Javascript}

Javascript is a scripting language; the instructions are written directly without a main class or a main function.
The program can be used side-by-side HTML, and therefore executed in a browser.

\lstinputlisting[language={[KB]Javascript}, style=codeStyle, firstline=6, lastline=9]{../codes/javascript/hello/helloworld.htm}

It can, also, be executed using nodeJs. 

\lstinputlisting[language={[KB]Javascript}, style=codeStyle]{../codes/javascript/hello/helloworld.js}

To execute it in command-line, you can use either one of these two commands
\begin{lstlisting}[style=shellStyle]
$ node helloworld.js
$ nodejs helloworld.js
\end{lstlisting}

\subsection{Lua}

Lua is a scripting language; the instructions are written directly without a main class or a main function.

\lstinputlisting[language={[KB]Lua}, style=codeStyle]{../codes/lua/hello/helloworld.lua}

To execute it in command-line:
\begin{lstlisting}[style=shellStyle]
$ lua helloworld.lua
\end{lstlisting}

\subsection{Perl}

Perl is a scripting language; the instructions are written directly without a main class or a main function.

\lstinputlisting[language={[KB]Perl}, style=codeStyle]{../codes/perl/hello/helloworld.pl}

To execute it in command-line:
\begin{lstlisting}[style=shellStyle]
$ perl helloworld.pl
\end{lstlisting}

\subsection{PHP}

PHP is a scripting language; the instructions are written directly without a main class or a main function.
A code in PHP is delimited by \keyword{<?php} and \keyword{?>}

\lstinputlisting[language={[KB]PHP}, style=codeStyle]{../codes/php/hello/helloworld.php}

To execute it in command-line:
\begin{lstlisting}[style=shellStyle]
$ php helloworld.php
\end{lstlisting}

\subsection{Python}

Python is a scripting language; the instructions are written directly without a main class or a main function.

\lstinputlisting[language={[KB]Python}, style=codeStyle]{../codes/python/hello/helloworld.py}

To execute it in command-line:
\begin{lstlisting}[style=shellStyle]
$ python helloworld.py
\end{lstlisting}

\subsection{Ruby}

Ruby is a scripting language; the instructions are written directly without a main class or a main function.

\lstinputlisting[language={[KB]Ruby}, style=codeStyle]{../codes/ruby/hello/helloworld.rb}

To execute it in command-line:
\begin{lstlisting}[style=shellStyle]
$ ruby helloworld.rb
\end{lstlisting}

\section{Functions}

Let's take the factorial function as an example. 
We implement the recursive version.  
The program, first, ask the user to introduce an integer number via the keyboard. 
Then it will call our function and print the result. 
We will print the code just for the function and not the entire program.
Please check the code to see how to read the keyboard.

\subsection{C++}

A function, mainly, contains a return type, its name and parameters. 
The language uses \keyword{\{\}} to define instructions blocks.
To return a value, the language uses the keyword \keyword{return}.
Statements end with a semicolon (\keyword{;}).

\lstinputlisting[language={[KB]C++}, linerange={12-16}, style=codeStyle]{../codes/cpp/hello/func.cpp}

\subsection{Java}

A function contains a modifier, a return type, its name and parameters.
It is always included in a class.
The language uses \keyword{\{\}} to define instructions blocks.
To return a value, the language uses the keyword \keyword{return}.
Statements end with a semicolon (\keyword{;}).

\lstinputlisting[language={[KB]Java}, linerange={5-5,15-22}, style=codeStyle]{../codes/java/src/hello/Func.java}

\subsection{Javascript}

A function starts with the keyword \keyword{function}, its name and parameters.
The language uses \keyword{\{\}} to define instructions blocks.
To return a value, the language uses the keyword \keyword{return}.
Statements end with a semicolon (\keyword{;}).

\lstinputlisting[language={[KB]Javascript}, style=codeStyle]{../codes/javascript/hello/func.js}

\subsection{Lua}

A function starts with the keyword \keyword{function}, its name and parameters.
All blocks are ended with the keyword \keyword{end}.
To return a value, the language uses the keyword \keyword{return}.
Each line contains at most one statement, but it can contains many separated by a semicolon.

\lstinputlisting[language={[KB]Lua}, linerange={1-6}, style=codeStyle]{../codes/lua/hello/func.lua}

\subsection{Perl}

A function starts with the keyword \keyword{sub} and its name.
The parameters can be recovered by shifting them from an array \keyword{@\_}.
The language uses \keyword{\{\}} to define instructions blocks.
To return a value, the language uses the keyword \keyword{return}.
Statements end with a semicolon (\keyword{;}).

\lstinputlisting[language={[KB]Perl}, linerange={7-13}, style=codeStyle]{../codes/perl/hello/func.pl}

\subsection{PHP}

A function starts with the keyword \keyword{function},its name and parameters.
The language uses \keyword{\{\}} to define statements blocks.
To return a value, the language uses the keyword \keyword{return}.
Statements end with a semicolon (\keyword{;}).

\lstinputlisting[language={[KB]PHP}, linerange={2-6}, style=codeStyle]{../codes/php/hello/func.php}

\subsection{Python}

A function starts with the keyword \keyword{def},its name and parameters.
The language uses indentation to define statements blocks.
To return a value, the language uses the keyword \keyword{return}.
Each line contains at most one statement, but simple statements can be separated by a semicolon.

\lstinputlisting[language={[KB]Python}, linerange={12-15}, style=codeStyle]{../codes/python/hello/func.py}

\subsection{Ruby}

A function starts with the keyword \keyword{def},its name and parameters.
All blocks are ended with the keyword \keyword{end}.
To return a value, you can either use the keyword \keyword{return} or put the value directly in a line.
Each line contains at most one statement, but simple statements can be separated by a semicolon.

\lstinputlisting[language={[KB]Ruby}, linerange={9-15}, style=codeStyle]{../codes/ruby/hello/func.rb}


\section{Entry point}

When a program is called from command-line, it searches for a function as entry-point. 
This is not the case for all languages; for instance, scripting languages execute without needing one. 
Nevertheless, they need a mechanism to recover the command-line parameters. 
Also, some languages, such as Python, afford a main function for entry-point lovers.

\subsection{C++}

In C++, main function is obligatory,  it can have parameters or not. 
The return type is an integer which indicates how the program exited. 
The normal state is 0; Otherwise, it indicates that there was an error during the process.
Contrary to C, in C++ you do not need to return explicitly.
In this case, the return value is an implicit 0.
\begin{lstlisting}[language={[KB]C++}, style=codeStyle]
int main()
int main(int argc, char * argv[])
\end{lstlisting}

When we want to recover command-line parameters, we have to use the main with arguments. 
\begin{itemize}
	\item argv (argument vector): is a table of pointers on chars (a table of strings).
	The first entry is the name of the program, and the others are parameters. 
	\item argc (argument count): is the size of that table
\end{itemize}

\lstinputlisting[language={[KB]C++}, linerange={1-17}, style=codeStyle]{../codes/cpp/hello/entry.cpp}

The header of our function \textbf{fact} is put ahead of the main, so we can use it. 
The compiler needs function declarations before the point of use (forward declaration). 
In our case, the implementation is after the main function.

\subsection{Java}

The main function is mandatory if we want an executable out of our class. 
It returns nothing (void) and have a list of string arguments. 
The main function must be always defined in the \keyword{public} class (which have the same name as the file). 
Also, it has to be \keyword{static}, so it can be executed without the need to create an instance of its containing class.

\lstinputlisting[language={[KB]Java}, style=codeStyle]{../codes/java/src/hello/Entry.java}

\subsection{Javascript}

In Javascript, the entire file is a big main, which can contain declarations (variables and functions) and statements.
The declarations are processed first, then the statements. 
This is why a function can be called before it is defined without the need to declare its header first. 
Being executed inside HTML, Javascript does not need arguments; But in case of Node.js, arguments are needed sometimes. 
The arguments passed by Node.js are stored in an array of strings \keyword{process.argv} which has \keyword{node} as its first elements, the name of the script as the second, and the rest are the command-line arguments.

\lstinputlisting[language={[KB]Javascript}, linerange={1-9}, style=codeStyle]{../codes/javascript/hello/entry.js}

\subsection{Lua}

Lua does not have a main function, since all the code can be considered as one. 
Also, functions must be defined before being called, because functions in lua are just values stored in a variable.
To recover command-line arguments, \keyword{arg} array is used. 

\lstinputlisting[language={[KB]Lua}, style=codeStyle]{../codes/lua/hello/entry.lua}

\subsection{Perl}

The whole file is a main in Perl, and functions can be called before being defined.
The command-line arguments are stored in a vector \keyword{\$ARGV}. 
In this code, we showed two methods to print the results.
The first one is the more interesting: it interpolates the function's call and the variable inside a string. 
The second one is a simple concatenation-based one. 

\lstinputlisting[language={[KB]Perl}, linerange={3-7}, style=codeStyle]{../codes/perl/hello/entry.pl}

\subsection{PHP}

PHP does not have a main function; the first line of code encountered in the file will be executed an on.
A function can be called before it was defined. 
To get command-line arguments, \keyword{\$argv} vector is used; Where the first element is the name of the script.

\lstinputlisting[language={[KB]PHP}, linerange={4-10}, style=codeStyle]{../codes/php/hello/entry.php}

\subsection{Python}

Python does not need a main function; The statements are executed one by one. 
But to prevent instructions from executing when the file is exported as a library, there is a main mechanism. 
A test if a built-in variable \keyword{\_\_name\_\_} contains the string \keyword{"\_\_main\_\_"} which means the module is being run directly from command-line.
Otherwise, this variable will contain the current module's name.

A function $ f1 $ can not be called before it was defined unless it is called inside another function $ f2 $. 
Even so, the second function $ f2 $ must be called after the definition of $ f1 $ and not before. 
In our example, if we define the factorial function after the \textbf{main} mechanism, we will get an error. 

To get the command-line arguments, we must import \keyword{sys} module and recover the vector \keyword{sys.argv}. 
The first element contains the name of the script and the rest contains the arguments.

\lstinputlisting[language={[KB]Python}, style=codeStyle]{../codes/python/hello/entry.py}

If you want to put the function in the end so badly, there is a solution. 
Just import the function from the same file under the main mechanism. 
In the following example, our file is called \textbf{entry.py}.

\begin{lstlisting}[language={[KB]Python}, style=codeStyle]
if __name__ == "__main__":
	from entry import fact
	...
def fact(i):
	...
	
\end{lstlisting}

\subsection{Ruby}

In Ruby, there is no main function, but there is a mechanism to prevent execution when the file is called as a module. 
There is a variable called \keyword{\_\_FILE\_\_} which contains the file name which is stored in \keyword{\$0} and \keyword{\$PROGRAM\_NAME}. 
The command-line arguments are stored in a variable \keyword{ARGV}.
Functions must be defined before being used, otherwise you will get an error.

\lstinputlisting[language={[KB]Ruby}, style=codeStyle]{../codes/ruby/hello/entry.rb}

For those who want it so badly to define a function after it is being called.
If you import the file using \keyword{require} keyword inside the main block, you will get a warning about the method being redefined. 
Another solution is to use the blocks: \keyword{BEGIN} or \keyword{END}. 
The former tells the interpreter to begin with this code, and the latter tells it to finish with it.

\begin{lstlisting}[language={[KB]Ruby}, style=codeStyle]
END {
	if __FILE__ == $0
		...
	end
}

def fact(i)
	...
end
\end{lstlisting}

\begin{lstlisting}[language={[KB]Ruby}, style=codeStyle]
if __FILE__ == $0
...
end

BEGIN {
	def fact(i)
		...
	end
}
\end{lstlisting}


\section{Exceptions}

\subsection{C++}

In C++, exceptions can be of any type. 
They are thrown using the keyword \keyword{throw}, and handled using \keyword{try} and \keyword{catch}.
You can throw a string as an exception.

\lstinputlisting[language={[KB]C++}, linerange={4-6}, style=codeStyle]{../codes/cpp/hello/except.cpp}

Then, capture it

\lstinputlisting[language={[KB]C++}, linerange={23-29}, style=codeStyle]{../codes/cpp/hello/except.cpp}


If you want to specify the type of the exception, here comes the OOP solution. 
You have to define new types by inheritance from \keyword{std::exception}. 
Then, redefine the function \keyword{what()} which affords the error message.

\lstinputlisting[language={[KB]C++}, linerange={6-16}, style=codeStyle]{../codes/cpp/hello/except2.cpp}

The exceptions can be thrown by creating a new instance of the defined classes. 

\lstinputlisting[language={[KB]C++}, linerange={19-20}, style=codeStyle]{../codes/cpp/hello/except2.cpp}

We catch them either by they parent's type \textbf{std::exception}, or by catching each type apart. 

\lstinputlisting[language={[KB]C++}, linerange={35-41}, style=codeStyle]{../codes/cpp/hello/except2.cpp}


\subsection{Java}

In Java, you can define exceptions by defining some classes which inherit from the class \keyword{Exception}. 
Or you can throw a new instance of the latter. 
The message in your new exception classes can be passed to the superclass, by the keyword \keyword{super} in their constructors. 
Also, you can redefine the function \keyword{getMessage()} as shown in our example.

\lstinputlisting[language={[KB]Java}, linerange={42-53}, style=codeStyle]{../codes/java/src/hello/Except.java}

The function which throw new instances of our exceptions, using the keyword \keyword{throw}, must declare that it have the potential to throw them in its header via the keyword \keyword{throws}. 

\lstinputlisting[language={[KB]Java}, linerange={33-35}, style=codeStyle]{../codes/java/src/hello/Except.java}

To capture these exceptions, you must use \keyword{try} and \keyword{catch}, or you can send them to an upper caller using \textbf{throws}. 
There is a keyword \keyword{finally} which is used to execute code in either cases: exception or not. 
Its benefit is to clean up before exiting; for instance: closing a file.

\lstinputlisting[language={[KB]Java}, linerange={20-29}, style=codeStyle]{../codes/java/src/hello/Except.java}

There are some other types of exceptions called \nameword{unchecked exceptions} which they don't force the programmer to capture them. 
You can define your own unchecked exceptions by inheriting from \keyword{RuntimeException}. 
In this case, you will not need to declare that your function is capable of throwing them. 
There are some built-in unchecked exceptions relative to arrays and number conversion as shown in this example.

\lstinputlisting[language={[KB]Java}, linerange={7-18}, style=codeStyle]{../codes/java/src/hello/Except.java}

\subsection{Javascript}

Using \keyword{throw} statement, you can throw an exception of any type: integer, string, object, etc. 

\lstinputlisting[language={[KB]Javascript}, linerange={21-23}, style=codeStyle]{../codes/javascript/hello/except.js}

The function's execution will stop and the control will be passed to the first catch block where it is being called.
If the error is not captured, the program will terminate.
To capture an exception, you need to call the function inside \keyword{try} and \keyword{catch}. 
You can add \keyword{finally} which will execute in both cases: failure or success.

\lstinputlisting[language={[KB]Javascript}, linerange={10-19}, style=codeStyle]{../codes/javascript/hello/except.js}

You can throw an error  using the class \keyword{error}

\begin{lstlisting}[language={[KB]Javascript}, style=codeStyle]
new Error([message[, fileName[, lineNumber]]])
\end{lstlisting}

\subsection{Lua}

As mentioned in Lua manual: you do not need error handling for most applications. 
But in case you want to throw an error, you can use the function \keyword{error}

\lstinputlisting[language={[KB]Lua}, linerange={1-3}, style=codeStyle]{../codes/lua/hello/except.lua}

Then, if you want to handle this error, you can use the function \keyword{pcall} to encapsulate your previous code. 
It will return a success indicator and the returned value from your function.
The value, here, is either the result from te function or the raised error. 

\lstinputlisting[language={[KB]Lua}, linerange={14-20}, style=codeStyle]{../codes/lua/hello/except.lua}

\subsection{Perl}

Perl does not have a good built-in mechanism for exceptions. 
This is why there are some perl modules designed specially for exception handling.
If you want to throw errors, you can use the keyword \keyword{die}. 

\lstinputlisting[language={[KB]Perl}, linerange={14-17}, style=codeStyle]{../codes/perl/hello/except.pl}

Using \keyword{eval}, you can capture the error into \keyword{\$@}.

\lstinputlisting[language={[KB]Perl}, linerange={6-12}, style=codeStyle]{../codes/perl/hello/except.pl}

\subsection{PHP}

In PHP, Exception handling is available in version 5 and greater.
To raise an exception, either you create an instance of the class \keyword{Exception} with a message as parameter, 
or you create custom exceptions by inheriting from that class.

\lstinputlisting[language={[KB]PHP}, linerange={4-14}, style=codeStyle]{../codes/php/hello/except.php}

To raise the exception, all you have to do is calling the keyword \keyword{throw} followed by an instance of the custom exception.

\lstinputlisting[language={[KB]PHP}, linerange={16-18}, style=codeStyle]{../codes/php/hello/except.php}

Then, to capture the exception, you can use \keyword{try} and \keyword{catch}.

\lstinputlisting[language={[KB]PHP}, linerange={30-39}, style=codeStyle]{../codes/php/hello/except.php}

From PHP5.5, a block of \keyword{finally} can be specified to execute code after normal execution or exception handling.

\subsection{Python}

The exceptions inherit from the class \keyword{Exception}.

\lstinputlisting[language={[KB]Python}, linerange={6-12}, style=codeStyle]{../codes/python/hello/except.py}

To throw an exception, use the keyword \keyword{raise}

\lstinputlisting[language={[KB]Python}, linerange={14-18}, style=codeStyle]{../codes/python/hello/except.py}

To capture these exceptions, use \keyword{try} and \keyword{except}. 
You have to capture the most specific exception first, then the generic ones. 
After \keyword{except} blocks, you can use a clean-up block: \keyword{finally}.

\lstinputlisting[language={[KB]Python}, linerange={28-32}, style=codeStyle]{../codes/python/hello/except.py}

There are some built-in exceptions you can use, such as this example.

\lstinputlisting[language={[KB]Python}, linerange={34-38}, style=codeStyle]{../codes/python/hello/except.py}

\subsection{Ruby}

In Ruby, you can throw an exception affording just a message which will create a new instance of \keyword{RuntimeError}. 
To throw an exception, you need to use the keyword \keyword{raise}

\lstinputlisting[language={[KB]Ruby}, linerange={1-6}, style=codeStyle]{../codes/ruby/hello/except.rb}

To handle the exception, you can use the keywords: 
\begin{itemize}
	\item begin: to begin the block
	\item rescue: to process the exception (you may have many of this)
	\item else: executes if there was no exception
	\item ensure: executes always in the end
	\item end: to close the block
\end{itemize}

\lstinputlisting[language={[KB]Ruby}, linerange={21-30}, style=codeStyle]{../codes/ruby/hello/except.rb}

You can specify custom classes to handle exceptions which can inherit from \keyword{StandardError}. 
It is a good practice to regroup these errors all together.

\lstinputlisting[language={[KB]Ruby}, linerange={1-12}, style=codeStyle]{../codes/ruby/hello/except2.rb}

To throw them, all you have to do is creating an instance after calling \keyword{raise}.

\lstinputlisting[language={[KB]Ruby}, linerange={16-19}, style=codeStyle]{../codes/ruby/hello/except2.rb}

You can capture them by their parent exception class \keyword{Exception}, or each by its class.

\lstinputlisting[language={[KB]Ruby}, linerange={37-40}, style=codeStyle]{../codes/ruby/hello/except2.rb}

\begin{discussion}

Before introducing oriented-object concepts of each language, how about a little comparison between our programming languages.
Table~\ref{tab-hello} represents a general comparison between OO languages based on these criteria:
\begin{itemize}
	\item \textbf{Type safety}: is the extent to which a programming language prevents type errors. 
	\item \textbf{Type expression}: represents the type declaration. If it is explicit, the programmer have to specify the type of a variable or a method. If it is implicit, the type is inferred based on how the variables are being used.
	\item \textbf{Type checking}: When the code is verified for type constrains, this can be happen in compile-time (static check) or run-time (dynamic check).
\end{itemize}


\extrarowsep = 0pt 

%	\doublerawsep = 1.5pt
\begin{longtabu} to \linewidth %
	{
		X[p]|[5pt white]
		X[2,p]|[5pt white]
		X[2,p]|[5pt white]
		X[2,p]|[5pt white]
	} %{llllllll} 
	%	\begin{longtabu} {p{1cm}p{1cm}p{5cm}p{1cm}p{1cm}p{2cm}p{2cm}p{1cm}}
	\caption{General comparison}%
	\label{tab-hello}\\
	
	%		\hline\hline
	\rowcolor{indigo}
	\rowfont{\bfseries\color{white}}
	{Language} &
	{Type safety} &
	{Type expression} &
	{Type checking} \\
	%		\hline\hline
	&&&\\
	\endfirsthead
	
	%		\hline\hline
	\rowcolor{indigo}
	\rowfont{\bfseries\color{white}}
	{Language} &
	{Type safety} &
	{Type expression} &
	{Type checking} \\
	%		\hline\hline
	&&&\\
	\endhead
	
	\taburowcolors{indigo!20!white .. black!10!white}
	
	{\bfseries\color{indigo}C++} & %Language
	Weak & 
	Explicit & 
	Static \\
	%		\hline
	
	{\bfseries\color{indigo}Java} & %Language
	Strong & 
	Explicit & 
	Static \\
	%		\hline
	
	{\bfseries\color{indigo}Javascript} & %Language
	Weak & 
	Implicit & 
	Dynamic \\
	%		\hline
	
	{\bfseries\color{indigo}Lua} & %Language
	Strong & 
	Implicit & 
	Dynamic \\
	%		\hline
	
	{\bfseries\color{indigo}Perl} & %Language
	& 
	Implicit & 
	Dynamic \\
	%		\hline
	
	{\bfseries\color{indigo}PHP} & %Language
	& 
	implicit/optional explicit & 
	Dynamic \\
	%		\hline
	
	{\bfseries\color{indigo}Python} & %Language
	Strong & 
	implicit/(v3.5+ optional explicit) & 
	Dynamic \\
	%		\hline
	
	{\bfseries\color{indigo}Ruby} & %Language
	Strong & 
	implicit & 
	Dynamic \\
	%		\hline
	
	%		\hline\hline
	
\end{longtabu}
	
\end{discussion}


%=====================================================================
\ifx\wholebook\relax\else
% \cleardoublepage
% \bibliographystyle{../use/ESIbib}
% \bibliography{../bib/RATstat}
	\end{document}
\fi
%=====================================================================