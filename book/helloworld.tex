%=====================================================================
\ifx\wholebook\relax\else
	\documentclass{KodeBook}
	\usepackage{amsmath,amssymb}             % AMS Math
\usepackage[utf8]{inputenc}
%\usepackage[T1]{fontenc}
\usepackage{hyperref}
%\usepackage[pdftex]{graphicx}
\usepackage{listings}
%\usepackage{listingsutf8}
%\usepackage{xcolor}
%\usepackage{times}
\usepackage{array}
%\usepackage{natbib}
%\usepackage[left=2.8cm,right=2.2cm,top=2.8cm,bottom=2.8cm,includefoot,includehead,headheight=13.6pt]{geometry}

%\bibliographystyle{ACM-Reference-Format-Journals}


\lstdefinestyle{codeStyle}{
  belowcaptionskip=1\baselineskip,
  breaklines=true,
  frame=L,
  xleftmargin=1.5cm,%\parindent,
  showstringspaces=false,
  basicstyle=\scriptsize\ttfamily\bfseries,
  keywordstyle=\bfseries\color{green!40!black},
  commentstyle=\itshape\color{purple!40!black},
  identifierstyle=\color{blue},
  stringstyle=\color{orange},
  backgroundcolor=\color{blue!10!white},
  lineskip=.2em,
  numbers=left
%  frameround=tttt
}

\lstdefinestyle{shellStyle}{
	belowcaptionskip=1\baselineskip,
	breaklines=true,
	frame=shadwbox,
	rulesepcolor=\color{blue},
	xleftmargin=1.25cm,
	basicstyle=\scriptsize\ttfamily\bfseries\color{green},
	backgroundcolor=\color{black},
	lineskip=.2em,
%	morekeywords={sudo},keywordstyle=\color{red},
	%  frameround=tttt
}



\hypersetup{
	colorlinks,
	urlcolor=blue
}

\renewcommand{\UrlFont}{\ttfamily\footnotesize}

	\begin{document}
	\chapter{Hello World}
\fi
%=====================================================================

%Introduction


\section{Getting started}

\subsection{C++}

C++ is a general-purpose compiled programming language which affords object-oriented paradigm among others. 
It was developed by \textbf{Bjarne Stroustrup} at Bell Labs since 1979.

To use C++ you have to install its compiler. 
The most used one is GNU g++\footnote{GNU g++: \url{gcc.gnu.org/}} which is distributed freely and available for most systems.

\subsubsection{Linux}
Simply install g++ on your system; on Ubuntu you, simply, type:
\begin{lstlisting}[style=shellStyle]
$ sudo apt-get update
$ sudo apt-get install g++
\end{lstlisting}

\subsubsection{Windows}

If you want to use g++ on Windows, you can install:
\begin{itemize}
	\item MinGW: \url{http://www.mingw.org} 
	\item CygWin: \url{http://www.cygwin.com} if you want to use Linux on Windows
\end{itemize}

You can use Microsoft Visual studio: \url{https://www.visualstudio.com/vs/visual-studio-express/}

Also, you can install Turbo C++ on \url{https://turboc.codeplex.com}

\subsection{Java}

Java is a general-purpose compiled programming language which is concurrent, class-based, object-oriented. 
It is intended to generate platform-independent programs. 
That is, the same compiled program called bytecodes will be able to execute in different machines and operating systems using a virtual machine. 
Developed by \textbf{James Gosling} at Sun Microsystems (acquired by Oracle Corporation) in 1995.

This is a detailed description on how to install Java \url{https://www.java.com/en/download/help/download_options.xml}.
Simply, to install Oracle JDK, download it from this link: \url{http://www.oracle.com/technetwork/java/javase/downloads/index.html}. 

As for OpenJDK, you can install it on Linux distributions following these instructions: \url{http://openjdk.java.net/install/}. 
On some distributions such as \textbf{Linux Mint}, it is installed by default.
For example, in Ubuntu you type:
\begin{lstlisting}[style=shellStyle]
$ sudo apt-get update
$ sudo apt-get install openjdk-8-jdk
\end{lstlisting}

It is recommended to use an IDE:
\begin{itemize}
	\item Eclipse: which is used in our case. \url{http://www.eclipse.org}
	\item IntelliJ IDEA: \url{https://www.jetbrains.com/idea/}
	\item NetBeans: \url{https://netbeans.org}
\end{itemize}

\subsection{Javascript}

JavaScript is an interpreted programming language for the web, which supports object-oriented paradigm. 
It is meant to be used as client-side programming language executed on different browser, but it can also execute on server side using NodeJs for example.
It was developed by Netscape Communications Corporation, and designed by \textbf{Brendan Eich} in 1995. 

Javascript can be used directly on any browser. 
Suppose we have a file called "func.js" containing a function "fact(n)" which calculates the factorial of a number. 
To use it, all you have to do is linking the file and calling the function in an HTML file:
\begin{lstlisting}[style=codeStyle]
<html>
  <head>
    <title>Functions</title>
      <script type="text/javascript" src="func.js">
      </script>
      <script type="text/javascript">
          function exec(){
              var n = document.getElementById("n").value;
              var res = "Fact(" + n + ")= ";
              res += fact(n);
              document.getElementById("result").innerHTML = res;
          }
      </script>
  </head>
    <body>
        Please enter an integer value: 
        <input type="number" id="n" />
        <button id="func" onclick="exec()">Factorial</button>
        <div id="result"></div>
    </body>
</html>
\end{lstlisting}


Or you can use \textbf{NodeJs} to execute it directly on the shell which is the case in our late examples. 
You can download it here: \url{https://nodejs.org/en/download/}. 
In case of some Linux distributions, you can install it from their repositories:
\begin{lstlisting}[style=shellStyle]
$ sudo apt-get update
$ sudo apt-get install nodejs
\end{lstlisting}

\subsection{Lua}

Lua is an interpreted programming language, which supports object-oriented paradigm. 
It was designed primarily for embedded systems.
Designed by Roberto Ierusalimschy, Luiz Henrique de Figueiredo, and Waldemar Celes in 1993.
It is mostly used as a scripting language for games development.

Download it from here: \url{http://lua-users.org/wiki/LuaBinaries}. 
In Ubuntu, you can install it by using its version number as:
\begin{lstlisting}[style=shellStyle]
$ sudo apt-get update
$ sudo apt-get install lua5.2
\end{lstlisting}


\subsection{Perl}

Lua is a general-purpose interpreted programming language, which supports oriented-object paradigm. 
It was developed by \textbf{Larry Wall} in 1987 to simplify report processing on Unix.
It is used as a CGI scripting language for web servers.

To install Perl, please refer to this page: \url{https://www.perl.org/get.html}. 
It is high probable that you have Perl installed on your Linux. 
As for Windows users, there are two projects: 
\begin{itemize}
	\item Strawberry Perl: \url{http://strawberryperl.com}
	\item ActiveState Perl: \url{http://www.activestate.com/activeperl/downloads}
\end{itemize}

\subsection{PHP}

PHP is a general-purpose server-side scripting language which supports oriented-object programming since version 3.0  and improved in version 5.0.
It was originally created by \textbf{Rasmus Lerdorf} in 1994. 

For a detailed description on how to install the server, please refer to this page: \url{http://php.net/manual/en/install.php}.
On Ubuntu, we just need the language:
\begin{lstlisting}[style=shellStyle]
$ sudo apt-get update
$ sudo apt-get install php
\end{lstlisting}
It will install a command-line interface (CLI) called "php"; used to execute a php file directly from the shell.

\subsection{Python}

Python is an interpreted general-purpose programming language, support multiple paradigms including OOP. 
It was created by \textbf{Guido van Rossum} and released in 1991.
It supports code readability as a design philosophy by using indentations as a mean of defining code blocks.

Refer to this page: \url{https://www.python.org/downloads/}. 
For Ubuntu, you can type:
\begin{lstlisting}[style=shellStyle]
$ sudo apt-get update
$ sudo apt-get install python
\end{lstlisting}
Or you can refer to this page for further information: \url{http://docs.python-guide.org/en/latest/starting/install3/linux/}.
You have to verify if it is already installed (Linux Mint has it by default).

\subsection{Ruby}

Ruby is an interpreted, object-oriented, general-purpose programming language. 
It was designed and developed by \textbf{Yukihiro Matsumoto} in 1990.

To install it, please refer to this page: \url{http://www.ruby-lang.org/en/documentation/installation/}. 
On Windows, you can use RubyInstaller: \url{https://rubyinstaller.org}.
On Linux systems, you can find it on their repositories, such as Ubuntu:
\begin{lstlisting}[style=shellStyle]
$ sudo apt-get update
$ sudo apt-get install ruby
\end{lstlisting}
You have to verify if it is already installed (Linux Mint has it by default).


\section{Languages features}

Before introducing oriented-object concepts of each language, how about a little comparison between our programming languages.

\subsection{Type safety}



\subsection{Type expression}


\subsection{Type checking}

\subsection{comparison}

\begin{table}
	\begin{tabular}{llll}
		\hline
		Language & Type safety & Type expression & Type checking \\
		\hline
		C++ & Weak & Explicit & Static \\
		\hline
		Java & Strong & Explicit & Static \\
		\hline
		Javascript & Weak & Implicit & Dynamic \\
		\hline
		Lua & Strong & Implicit & Dynamic \\
		\hline
		Perl & & Implicit & Dynamic \\
		\hline
		PHP & & implicit/optional explicit & Dynamic \\
		\hline
		Python & Strong & implicit/(v3.5+ optional explicit) & Dynamic \\
		\hline
		Ruby & Strong & implicit & Dynamic \\
		\hline
	\end{tabular}
\end{table}



\section{Hello world}

Let's start with a Hello World introductory set of codes. 


\subsection{C++}

A C++ program always has a main function, which represents the interface with the operating system. 
When you call a program (from shell for example), you actually calling this function. 
%
\lstinputlisting[language=C++, style=codeStyle]{../cpp/helloworld.cpp}
%
In our example, the function "main" doesn't have any arguments. 
If we want to pass some arguments to our program, we have to add a list of arguments to the "main" function
%
\begin{lstlisting}[language=C++, style=codeStyle]
int main(int argc, char * argv[])
\end{lstlisting}
Where:
\begin{itemize}
	\item argv: is a table of pointers on chars (a table of strings).
	\item argc: is the size of that table
\end{itemize}


\subsection{Java}

\lstinputlisting[language=Java, style=codeStyle]{../java/src/HelloWorld.java}

\subsection{Javascript}

\lstinputlisting[style=codeStyle, firstline=6, lastline=9]{../javascript/helloworld.htm}

\subsection{Lua}

\lstinputlisting[language={[5.2]Lua}, style=codeStyle]{../lua/helloworld.lua}

\subsection{Perl}

\lstinputlisting[language=Perl, style=codeStyle]{../perl/helloworld.pl}

\subsection{PHP}

\lstinputlisting[language=PHP, style=codeStyle]{../php/helloworld.php}

\subsection{Python}

\lstinputlisting[language=Python, style=codeStyle]{../python/helloworld.py}

\subsection{Ruby}

\lstinputlisting[language=Ruby, style=codeStyle]{../ruby/helloworld.rb}

\section{Functions}

Let take the factorial function as an example. 

\subsection{C++}

\lstinputlisting[language=C++, style=codeStyle]{../cpp/func.cpp}
%\lstinputlisting[language=Python, firstline=37, lastline=45]{source_filename.py}

\subsection{Java}

\lstinputlisting[language=Java, style=codeStyle]{../java/src/Func.java}

\subsection{Javascript}

\lstinputlisting[style=codeStyle]{../javascript/func.js}

\subsection{Lua}

\lstinputlisting[language={[5.2]Lua}, style=codeStyle]{../lua/func.lua}

\subsection{Perl}

\lstinputlisting[language=Perl, style=codeStyle]{../perl/func.pl}

\subsection{PHP}

\lstinputlisting[language=PHP, style=codeStyle]{../php/func.php}

\subsection{Python}

\lstinputlisting[language=Python, style=codeStyle]{../python/func.py}

\subsection{Ruby}

\lstinputlisting[language=Ruby, style=codeStyle]{../ruby/func.rb}


\section{Entry point}

Let take the factorial function as an example. 


\subsection{C++}

In C++, main function is obligatory. 
The language can accept many forms: 



\lstinputlisting[language=C++, style=codeStyle]{../cpp/func.cpp}
%\lstinputlisting[language=Python, firstline=37, lastline=45]{source_filename.py}

%\subsection{Java}
%
%\lstinputlisting[language=Java, style=codeStyle]{../java/src/Func.java}
%
%\subsection{Javascript}
%
%\lstinputlisting[style=codeStyle]{../javascript/func.js}
%
%\subsection{Lua}
%
%\lstinputlisting[language={[5.2]Lua}, style=codeStyle]{../lua/func.lua}
%
%\subsection{Perl}
%
%\lstinputlisting[language=Perl, style=codeStyle]{../perl/func.pl}
%
%\subsection{PHP}
%
%\lstinputlisting[language=PHP, style=codeStyle]{../php/func.php}
%
%\subsection{Python}
%
%\lstinputlisting[language=Python, style=codeStyle]{../python/func.py}
%
%\subsection{Ruby}
%
%\lstinputlisting[language=Ruby, style=codeStyle]{../ruby/func.rb}

\section{Exceptions}

\subsection{C++}

In C++, exceptions can be of any type. 
They are thrown using the keyword \textbf{throw}, and handled using \textbf{try} and \textbf{catch}.

\lstinputlisting[language=C++, style=codeStyle]{../cpp/except.cpp}

If you want to specify the type of the exception, here comes the OOP solution. 
You have to define new types by inheritance from \textbf{std::exception}. 
Then, redefine the function \textbf{what()}.

\lstinputlisting[language=C++, firstline=3, lastline=16, style=codeStyle]{../cpp/except2.cpp}

The exceptions can be thrown by creating a new instance of the defined classes. 

\lstinputlisting[language=C++, firstline=19, lastline=22, style=codeStyle]{../cpp/except2.cpp}

We catch them either by they parent's type, or by catching each type apart. 

\lstinputlisting[language=C++, firstline=43, lastline=45, style=codeStyle]{../cpp/except2.cpp}

%\lstinputlisting[language=Python, firstline=37, lastline=45]{source_filename.py}
%
%\subsection{Java}
%
%\lstinputlisting[language=Java, style=codeStyle]{../java/src/Func.java}
%
%\subsection{Javascript}
%
%\lstinputlisting[style=codeStyle]{../javascript/func.js}
%
%\subsection{Lua}
%
%\lstinputlisting[language={[5.2]Lua}, style=codeStyle]{../lua/func.lua}
%
%\subsection{Perl}
%
%\lstinputlisting[language=Perl, style=codeStyle]{../perl/func.pl}
%
%\subsection{PHP}
%
%\lstinputlisting[language=PHP, style=codeStyle]{../php/func.php}
%
%\subsection{Python}
%
%\lstinputlisting[language=Python, style=codeStyle]{../python/func.py}
%
%\subsection{Ruby}
%
%\lstinputlisting[language=Ruby, style=codeStyle]{../ruby/func.rb}




%=====================================================================
\ifx\wholebook\relax\else
% \cleardoublepage
% \bibliographystyle{../use/ESIbib}
% \bibliography{../bib/RATstat}
	\end{document}
\fi
%=====================================================================