\usepackage{amsmath,amssymb}             % AMS Math
\usepackage[utf8]{inputenc}
%\usepackage[T1]{fontenc}

%\usepackage[pdftex]{graphicx}

%\usepackage{listingsutf8}
%\usepackage{xcolor}
%\usepackage{times}
\usepackage{array}
\usepackage{natbib}
\usepackage{lscape}%to flip tables in a page
\usepackage{pdflscape}
\usepackage{longtable}
\usepackage{tabu}
\usepackage{wrapfig}

%\usepackage[english]{babel}

\bibliographystyle{engdnat}%unsrtnat, plainnat

%\usepackage{pgf-umlcd}




\hypersetup{
	pdfkeywords={OOP; Programming; Coding},
	pdfsubject={Computing; Computer programming}
}

\renewcommand{\UrlFont}{\ttfamily\footnotesize}

\DeclareAcronym{oop}{
	short = OOP ,
	long  = Object-oriented programming ,
	class = abbrev
}

\DeclareAcronym{oo}{
	short = OO ,
	long  = Object-oriented ,
	class = abbrev
}

\DeclareAcronym{ecma}{
	short = ECMA ,
	long  = European Computer Manufacturers Association ,
	class = abbrev
}

\DeclareAcronym{cli}{
	short = CLI ,
	long  = Command-Line Interface ,
	class = abbrev
}

\DeclareAcronym{cgi}{
	short = CGI ,
	long  = Common Gateway Interface ,
	class = abbrev
}

\DeclareAcronym{es6}{
	short = ES6 ,
	long  = ECMAScript 2015 ,
	class = abbrev
}

\DeclareAcronym{es5}{
	short = ES5 ,
	long  = ECMAScript 5 ,
	class = abbrev
}

%\makeglossaries

%\newacronym{oop}{OOP}{Object-oriented programming}